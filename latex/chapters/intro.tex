\chapter*{مقدمه}
\addcontentsline{toc}{chapter}{مقدمه} 
در دنیای مدرن، بازارهای مالی به‌عنوان یکی از پیچیده‌ترین و پویا‌ترین سیستم‌های اقتصادی شناخته می‌شوند که نقش مهمی در توسعه اقتصاد جهانی ایفا می‌کنند. تغییرات نوسانی قیمت‌ها در این بازارها، به خصوص در زمینه رمز ارزها \footnote{\lr{Cryptocurrencies}}، به دلیل ماهیت غیرمرکزی، بدون نهاد ناظر بودن و حساس به عوامل مختلف، چالش‌برانگیزتر از سایر دارایی‌های مالی است. رشد سریع رمز ارزها از اوایل دهه ۲۰۱۰ و ظهور بیت‌کوین \footnote{\lr{Bitcoin}}  به عنوان اولین رمز ارز دیجیتال، باعث شده است که این دارایی‌های مجازی به‌عنوان یکی از جذاب‌ترین و همچنین ریسکی ترین گزینه‌های سرمایه‌گذاری در دهه‌های اخیر شناخته شوند. با وجود این موارد، عدم قطعیت بالا در قیمت‌های رمز ارزها، به دلیل تأثیرپذیری آن‌ها از عوامل متعددی مانند نوسانات بازار، اخبار، رفتار سرمایه‌گذاران، و حتی رویدادهای سیاسی و اجتماعی، سرمایه‌گذاران و تصمیم‌گیرندگان را ملزم به داشتن ابزارهای دقیق‌تر برای پیش‌بینی قیمت‌ها می‌کند.


مدل‌سازی و پیش‌بینی قیمت رمز ارزها یکی از موضوعات مرکزی در تحقیقات مربوط به هوش مصنوعی \footnote{\lr{Artificial Intelligence (AI)}} و یادگیری ماشین \footnote{\lr{Machine Learning}} در دهه‌های اخیر شده است. تکنیک‌های مدرن داده‌کاوی، شبکه‌های عصبی عمیق \footnote{\lr{Deep Neural Networks}} و الگوریتم‌های یادگیری ماشین پیشرفته، فرصت‌های فراوانی را برای تحلیل و پیش‌بینی رفتار بازارهای مالی فراهم کرده‌اند. با این حال، پیچیدگی ساختاری داده‌های مرتبط با رمز ارزها، که شامل نوسانات زمانی، تعداد زیادی ویژگی و عوامل اثر گذار، و ارتباطات غیرخطی بین متغیرها است، باعث شده تا پژوهشگران و فعالین بازار به دنبال روش‌های نوین‌تر و دقیق‌تر برای مدل‌سازی قیمت‌ها بروند.


این پایان‌نامه به بررسی و توسعه یک مدل پیش‌بینی قیمت رمز ارزها مبتنی بر روش‌های بسامد داده‌ها، شبکه‌های عصبی و هوش مصنوعی می‌پردازد. در این پژوهش، سعی شده است تا از ترکیب روش‌های تحلیل بسامدی داده‌ها، که قادر به شناسایی الگوهای دوره‌ای و غیرخطی در داده‌ها هستند، و شبکه‌های عصبی عمیق، که برای یادگیری الگوهای پیچیده و غیرخطی مناسب هستند، بهره گرفته شود. هدف اصلی این پژوهش، توسعه یک مدل ترکیبی است که بتواند با دقت بالاتری نسبت به روش‌های موجود، قیمت‌های آتی رمز ارزها را پیش‌بینی کند.

مدل پیشنهادی این پژوهش، از تحلیل بسامدی داده‌ها برای شناسایی الگوهای دوره‌ای در قیمت‌های روزانه، ساعتی و دقیقه‌ای رمز ارزها استفاده می‌کند. این الگوها سپس به‌عنوان ورودی برای یک شبکه عصبی عمیق ارائه می‌شوند که مسئول یادگیری الگوهای پیچیده و غیرخطی در داده‌هاست. از طرف دیگر، استفاده از هوش مصنوعی در این مدل، به صورت مستقیم به بهبود عملکرد پیش‌بینی کمک می‌کند و امکان درنظرگرفتن متغیرهای جانبی مانند اخبار، احساسات سرمایه‌گذاران و رویدادهای سیاسی را فراهم می‌آورد.

این پژوهش، با توجه به چالش‌های موجود در پیش‌بینی قیمت رمز ارزها، به دنبال پاسخ‌گویی به سؤالات اساسی زیر است:

\begin{itemize}
	\item چگونه می‌توان از ترکیب روش‌های بسامدی داده‌ها و شبکه‌های عصبی برای پیش‌بینی قیمت رمز ارزها استفاده کرد؟
	\item چه الگوهایی در داده‌های قیمتی رمز ارزها وجود دارد که می‌توانند توسط تحلیل بسامدی شناسایی شوند؟
	\item چگونه مدل پیشنهادی می‌تواند دقت پیش‌بینی را نسبت به روش‌های سنتی بهبود بخشند؟
\end{itemize}

در ادامه، این پایان‌نامه شامل بررسی ادبیات پژوهشی در زمینه پیش‌بینی قیمت رمز ارزها، توضیح روش‌شناسی پژوهش، تشریح مدل پیشنهادی، ارزیابی عملکرد مدل با استفاده از داده‌های واقعی، و نتایج نهایی می‌شود. 
این پژوهش، با توجه به اهمیت رشد بازار رمز ارزها و نیاز به ابزارهای پیشرفته پیش‌بینی، به یک پاسخ عملی و علمی برای چالش‌های موجود در این زمینه می‌پردازد. هدف نهایی این پژوهش، ارائه یک مدل قابل اعتماد برای کمک به سرمایه‌گذاران، تصمیم‌گیرندگان و مدیران ریسک در بازارهای رمز ارز است.