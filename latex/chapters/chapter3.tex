\chapter{نتایج}
\label{chap:results}

\section{مقدمه}
فصل سوم این پایان‌نامه به ارائه و تحلیل نتایج حاصل از طراحی و اعتبارسنجی مدل پیش‌بینی قیمت بیت‌کوین (BTC) اختصاص دارد. این مدل، همان‌طور که در فصل دوم تشریح شد، ترکیبی از تحلیل بسامد داده‌ها، شبکه‌های عصبی (به‌ویژه شبکه‌های حافظه کوتاه‌مدت بلندمدت یا LSTM) و تکنیک‌های هوش مصنوعی است. هدف این فصل، بررسی عملکرد مدل در پیش‌بینی قیمت بیت‌کوین با استفاده از داده‌های تاریخی از ژانویه ۲۰۱۸ تا دسامبر ۲۰۲۴ است که از پلتفرم‌های معتبر مانند \lr{CoinMarketCap} و \lr{Binance} جمع‌آوری شده‌اند. 

نتایج این فصل بر اساس داده‌های روزانه بیت‌کوین ارائه شده‌اند، اگرچه داده‌های با فرکانس بالا (مانند داده‌های دقیقه‌ای) نیز در تحلیل‌های اولیه مورد استفاده قرار گرفته‌اند. معیارهای ارزیابی عملکرد شامل میانگین خطای مطلق (MAE)، ریشه میانگین مربعات خطا (RMSE) و ضریب تعیین (\(R^2\)) هستند که برای سنجش دقت مدل به کار رفته‌اند. علاوه بر این، تحلیل‌های بسامدی مانند تبدیل فوریه سریع (FFT) و تبدیل موجک برای شناسایی الگوهای دوره‌ای و روندهای پنهان در داده‌ها انجام شده‌اند.

این فصل به بخش‌های زیر تقسیم شده است:
\begin{itemize}
	\item آماده‌سازی داده‌ها و تحلیل اولیه
	\item نتایج تحلیل بسامد داده‌ها
	\item عملکرد مدل‌های شبکه عصبی (LSTM)
	\item ارزیابی کلی مدل و مقایسه با روش‌های مرجع
	\item تحلیل محدودیت‌ها و چالش‌ها
	\item نتیجه‌گیری و بحث
\end{itemize}

\section{آماده‌سازی داده‌ها و تحلیل اولیه}
\label{sec:data_preparation}

\subsection{جمع‌آوری داده‌ها}
داده‌های مورد استفاده در این پژوهش شامل سری‌های زمانی قیمت روزانه بیت‌کوین از ژانویه ۲۰۱۸ تا دسامبر ۲۰۲۴ است. این داده‌ها از طریق کتابخانه \texttt{yfinance} از پلتفرم \lr{Yahoo Finance} استخراج شدند، که به عنوان یک منبع معتبر برای داده‌های تاریخی رمز ارزها شناخته می‌شود. مجموعه داده شامل 2557 نقطه داده (روزانه) است که قیمت بسته شدن \lr{(Close Price)} بیت‌کوین، اتریوم و  را در بر می‌گیرد. خلاصه‌ای از داده‌ها در \ref{tab:btc_describe} آمده است. داده‌های بلاک چین از وب سایت \href{https://www.blockchain.com/}{بلاک چین}\footnote{با استفاده از \lr{Api} که از طریق \href{https://api.blockchain.info/}{اینجا} قابل ارزیابی است.}.
 
 






\begin{table}[H]

	\begin{tabular}{|l|l|l|l|l|l|}

		\hline
		%Price & Close & High & Low & Open & Volume \\
		
		قیمت& بسته شدن& بالاترین&پایین ترین &باز شدن &حجم \\
		\hline
		count & 2556  & 2556  & 2556  & 2556  & 2556 \\
		mean & 28060/07 & 28650/84 & 27380/99 & 28031/10 & 26953718376/74 \\
		std & 22106/14 & 22560/37 & 21575/17 & 22073/16 & 19656354619/1 \\
		min & 3236/76 & 3275/37 & 3191/30 & 3236/27 & 2923670016  \\
		25\% & 8975/98 & 9202/10 & 8792/07 & 8940/94 & 13808137181  \\
		50\% & 22185/37 & 22598/98 & 21453/02 & 22013/65 & 24054431603  \\
		75\% & 42840/40 & 43575/49 & 41906/51 & 42807/78 & 35360104401  \\
		max & 106140/60 & 108268/44 & 105291/73 & 106147/29 & 350967941479 \\
		\hline
	
	\end{tabular}
\caption{خلاصه‌ی آماری داده‌های قیمت بیت کوین}\label{tab:btc_describe}
\end{table}

\begin{table}[H]
	\begin{tabular}{|l|l|l|l|l|}
		\hline 
		& ارتفاع بلاک & تعداد تراکنش &  حجم تراکنش‌ها به دلار & حجم تراکنش‌ها به بیت کوین \\
		\hline
	 	%& BlockHeight & TxCount & TxVolumeUSD & TxVolumeBTC \\
		count & 2556 & 2556 & 2556  & 2556 \\
		mean & 501277/50 & 242325/00 & 1418387677927/70 & 85716151/24 \\
		std & 737/99 & 0/000000 & 0/059582 & 0/000003 \\
		min & 500000/00 & 242325/00 & 1418387677927/70 & 85716151/24 \\
		25\% & 500638/75 & 242325/00 & 1418387677927/70 & 85716151/24 \\
		50\% & 501277/50 & 242325/00 & 1418387677927/70 & 85716151/24 \\
		75\% & 501916/25 & 242325/00 & 1418387677927/70 & 85716151/24 \\
		max & 502555/00 & 242325/00 & 1418387677927/70 & 85716151/24 \\
		\hline
	\end{tabular}
	
	\footnotetext{حجم به دلار حاصل ضرف تعداد بیت کوین در قیمت بسته شدن همان زمان است}
\caption{خلاصه‌ی آماری داده‌های بلاک چین بیت کوین}\label{tab:btc_blockchain_describe}

\end{table}


\subsection{پیش‌پردازش داده‌ها}
پیش‌پردازش داده‌ها شامل مراحل زیر بود:
\begin{itemize}
	\item \textbf{پاک‌سازی داده‌ها}: مقادیر گمشده با استفاده از روش پر کردن رو به جلو (Forward Fill) پر شدند. این روش به دلیل حفظ پیوستگی سری زمانی انتخاب شد.
	\item \textbf{نرمال‌سازی}: داده‌ها با استفاده از مقیاس‌بندی Min-Max به بازه [0,1] نرمال‌سازی شدند تا عملکرد مدل‌های یادگیری عمیق بهبود یابد.
	\item \textbf{تقسیم داده‌ها}: داده‌ها به سه مجموعه تقسیم شدند: 70\% برای آموزش (1789 روز)، 15\% برای اعتبارسنجی (383 روز) و 15\% برای آزمایش (385 روز). این تقسیم‌بندی به منظور ارزیابی تعمیم‌پذیری مدل انجام شد.
\end{itemize}

\subsection{تحلیل اولیه داده‌ها}
تحلیل اولیه داده‌ها نشان داد که سری زمانی قیمت بیت‌کوین دارای نوسانات قابل‌توجهی است، به‌ویژه در سال‌های 2021 و 2022 که بازار رمز ارزها دوره‌های صعودی و نزولی شدیدی را تجربه کرد. شکل \ref{fig:btc_price} نمودار قیمت روزانه بیت‌کوین را در بازه زمانی مورد مطالعه نشان می‌دهد. این نمودار نشان‌دهنده روند کلی قیمت بیت‌کوین است که شامل افت شدید در سال 2018، رشد قابل‌توجه در سال‌های 2020 و 2021، و نوسانات متعاقب در سال‌های 2022 تا 2024 است.

\begin{figure}[H]
	\centering
	\includegraphics[width=0.8\textwidth]{btc_price_plot}
	\caption{نمودار قیمت روزانه بیت‌کوین (ژانویه ۲۰۱۸ تا دسامبر ۲۰۲۴)}
	\label{fig:btc_price}
\end{figure}

این الگوهای قیمتی با یافته‌های \cite{urquhart2016inefficiency} هم‌خوانی دارد که نشان داد بازار رمز ارزها به شدت تحت تأثیر عوامل خارجی مانند اخبار و احساسات بازار قرار دارد. همچنین، نوسانات شدید در بازه‌های زمانی کوتاه‌مدت، پیچیدگی پیش‌بینی قیمت‌ها را افزایش داده است، که ضرورت استفاده از روش‌های پیشرفته مانند تحلیل بسامد و یادگیری عمیق را تأیید می‌کند.

\section{نتایج تحلیل بسامد داده‌ها}
\label{sec:frequency_results}

\subsection{تبدیل فوریه سریع (FFT)}
تحلیل بسامد داده‌ها با استفاده از تبدیل فوریه سریع (FFT) انجام شد تا الگوهای دوره‌ای در سری زمانی قیمت بیت‌کوین شناسایی شوند. تبدیل فوریه به صورت زیر تعریف می‌شود:
\begin{equation}
	X(k) = \sum_{n=0}^{N-1} x(n) e^{-j2\pi kn/N}
\end{equation}
که در آن \(X(k)\) طیف فرکانسی، \(x(n)\) سری زمانی قیمت، و \(N\) تعداد نمونه‌ها است. این روش به شناسایی فرکانس‌های غالب در داده‌ها کمک می‌کند که می‌توانند نشان‌دهنده چرخه‌های بازار یا روندهای پنهان باشند.

شکل \ref{fig:fft_spectrum} طیف فرکانسی قیمت روزانه بیت‌کوین را نشان می‌دهد. همان‌طور که مشاهده می‌شود، یک پیک قابل‌توجه در فرکانس‌های نزدیک به صفر (فرکانس‌های پایین) وجود دارد که نشان‌دهنده وجود روندهای بلندمدت در داده‌ها است. این پیک با مقیاس Amplitude در حدود \(7 \times 10^7\) نشان‌دهنده قدرت قابل‌توجه این مؤلفه فرکانسی است. فرکانس‌های پایین معمولاً با چرخه‌های بازار (مانند دوره‌های صعودی و نزولی چندساله) مرتبط هستند، در حالی که فرکانس‌های بالاتر (نزدیک به 0.5) نشان‌دهنده نوسانات کوتاه‌مدت هستند که در این مورد کمتر برجسته‌اند.

\begin{figure}[H]
	\centering
	\includegraphics[width=0.8\textwidth]{fft_spectrum}
	\caption{طیف فرکانسی قیمت روزانه بیت‌کوین با استفاده از FFT}
	\label{fig:fft_spectrum}
\end{figure}

این نتایج با مطالعات \cite{nguyen2021wavelet} هم‌خوانی دارد که نشان داد تحلیل بسامد می‌تواند الگوهای دوره‌ای در داده‌های رمز ارزها را شناسایی کند. پیک‌های فرکانسی پایین در این تحلیل نشان‌دهنده وجود چرخه‌های بلندمدت هستند که می‌توانند با عوامل اقتصادی کلان یا رویدادهای بازار (مانند هاوینگ بیت‌کوین) مرتبط باشند. با این حال، کاهش سریع دامنه (Amplitude) با افزایش فرکانس نشان‌دهنده این است که نوسانات کوتاه‌مدت (مانند تغییرات روزانه یا هفتگی) کمتر در سری زمانی غالب هستند، که ممکن است به دلیل نویز بالا در داده‌های رمز ارزها باشد، همان‌طور که توسط \cite{fry2018market} اشاره شده است.

\subsection{تبدیل موجک}
برای تجزیه داده‌ها به مقیاس‌های مختلف و شناسایی الگوهای کوتاه‌مدت و بلندمدت، از تبدیل موجک استفاده شد. تبدیل موجک به صورت زیر تعریف می‌شود:
\begin{equation}
	W(s,\tau) = \int_{-\infty}^{\infty} x(t) \psi^*\left(\frac{t-\tau}{s}\right) dt
\end{equation}
که در آن \(\psi\) تابع موجک، \(s\) مقیاس، و \(\tau\) جابجایی زمانی است. در این پژوهش، از موجک \texttt{db4} (Daubechies 4) با سطح تجزیه 4 استفاده شد تا داده‌ها به مؤلفه‌های مختلف مقیاس تجزیه شوند.

شکل \ref{fig:wavelet_transform} نتایج تجزیه موجک را در پنج سطح (یک مؤلفه تقریبی و چهار مؤلفه جزئی) نشان می‌دهد. مؤلفه تقریبی (Wavelet Coefficient 0) روند کلی و بلندمدت سری زمانی را نشان می‌دهد که با الگوهای مشاهده‌شده در شکل \ref{fig:btc_price} هم‌خوانی دارد. این مؤلفه نشان‌دهنده افزایش کلی قیمت بیت‌کوین از سال 2020 به بعد است. مؤلفه‌های جزئی (Wavelet Coefficients 1 تا 4) نوسانات در مقیاس‌های مختلف را نشان می‌دهند، از مقیاس‌های بزرگ‌تر (Coefficient 1) تا مقیاس‌های کوچک‌تر (Coefficient 4).

\begin{figure}[H]
	\centering
	\includegraphics[width=0.8\textwidth]{wavelet_transform}
	\caption{تجزیه موجک قیمت بیت‌کوین در مقیاس‌های مختلف}
	\label{fig:wavelet_transform}
\end{figure}

مؤلفه‌های جزئی نشان‌دهنده نوسانات کوتاه‌مدت هستند که در دوره‌های پرنوسان بازار (مانند سال‌های 2021 و 2022) شدت بیشتری دارند. به عنوان مثال، Coefficient 4 که نشان‌دهنده نوسانات با فرکانس بالا است، در بازه‌های زمانی با تغییرات قیمتی شدید (مانند اوج قیمت در سال 2021 و افت متعاقب آن در سال 2022) دامنه بیشتری دارد. این نتایج با یافته‌های \cite{lahmiri2018chaos} مطابقت دارد که از تجزیه موجک برای بهبود وضوح سیگنال در داده‌های بیت‌کوین استفاده کرد و نشان داد که این روش می‌تواند نویز را از روندهای اصلی جدا کند.

تحلیل موجک همچنین به شناسایی دوره‌های زمانی با تغییرات ناگهانی کمک کرد. به عنوان مثال، افزایش دامنه در Coefficient 4 در بازه‌های زمانی 2021 و 2022 با رویدادهای بازار مانند سقوط بازار پس از اوج قیمت بیت‌کوین در نوامبر 2021 (حدود 69000 دلار) و تأثیرات نظارتی در سال 2022 هم‌خوانی دارد. این یافته‌ها نشان‌دهنده توانایی تبدیل موجک در ثبت دینامیک‌های غیرخطی و غیرایستای سری‌های زمانی است، که با مطالعات \cite{mallat1999wavelet} و \cite{percival2000wavelet} هم‌راستا است.

\section{عملکرد مدل‌های شبکه عصبی (LSTM)}
\label{sec:nn_results}

\subsection{آموزش و پیش‌بینی با شبکه LSTM}
شبکه‌های حافظه کوتاه‌مدت بلندمدت (LSTM) برای پیش‌بینی قیمت بیت‌کوین طراحی و آموزش داده شدند. این مدل شامل دو لایه LSTM (هر کدام با 50 واحد)، دو لایه Dropout (برای جلوگیری از بیش‌برازش) و دو لایه Dense بود. مدل با استفاده از داده‌های آموزشی (70\% از داده‌ها) و اعتبارسنجی (15\% از داده‌ها) آموزش داده شد و سپس بر روی مجموعه آزمایش (15\% باقی‌مانده) ارزیابی شد. طول دنباله (Sequence Length) برای مدل 60 روز انتخاب شد، به این معنا که هر پیش‌بینی بر اساس 60 روز داده قبلی انجام می‌شود.

نتایج پیش‌بینی مدل LSTM در شکل \ref{fig:lstm_predictions} نشان داده شده است. این نمودار مقایسه‌ای بین قیمت‌های واقعی و پیش‌بینی‌شده بیت‌کوین در مجموعه آزمایش (385 روز آخر) ارائه می‌دهد. همان‌طور که مشاهده می‌شود، مدل LSTM به طور کلی روند قیمت‌ها را با دقت بالایی دنبال می‌کند، به‌ویژه در دوره‌های با تغییرات تدریجی (مانند روند صعودی در اواخر مجموعه آزمایش). با این حال، در دوره‌های با نوسانات شدید (مانند افت‌های ناگهانی قیمت در اوایل مجموعه آزمایش)، مدل کمی از واقعیت عقب می‌ماند و تمایل به پیش‌بینی‌های هموارتر دارد.

\begin{figure}[H]
	\centering
	\includegraphics[width=0.8\textwidth]{lstm_predictions}
	\caption{مقایسه قیمت‌های واقعی و پیش‌بینی‌شده بیت‌کوین توسط مدل LSTM}
	\label{fig:lstm_predictions}
\end{figure}

\subsection{معیارهای عملکرد}
برای ارزیابی عملکرد مدل، از معیارهای زیر استفاده شد:
\begin{itemize}
	\item \textbf{میانگین خطای مطلق (MAE)}:
	\begin{equation}
		\text{MAE} = \frac{1}{n} \sum_{i=1}^n |y_i - \hat{y}_i|
	\end{equation}
	\item \textbf{ریشه میانگین مربعات خطا (RMSE)}:
	\begin{equation}
		\text{RMSE} = \sqrt{\frac{1}{n} \sum_{i=1}^n (y_i - \hat{y}_i)^2}
	\end{equation}
	\item \textbf{ضریب تعیین (\(R^2\))}:
	\begin{equation}
		R^2 = 1 - \frac{\sum (y_i - \hat{y}_i)^2}{\sum (y_i - \bar{y})^2}
	\end{equation}
\end{itemize}

نتایج ارزیابی مدل در جدول \ref{tab:lstm_performance} ارائه شده است. این جدول نشان می‌دهد که مدل در مجموعه آموزشی عملکرد بسیار خوبی دارد (MAE = 0.008، RMSE = 0.015، \(R^2\) = 0.92)، و این عملکرد در مجموعه‌های اعتبارسنجی و آزمایش نیز حفظ می‌شود، اگرچه با کاهش اندکی همراه است (MAE = 0.012 و \(R^2\) = 0.88 در مجموعه آزمایش). این نتایج نشان‌دهنده تعمیم‌پذیری خوب مدل است، اگرچه کاهش \(R^2\) در مجموعه آزمایش ممکن است به دلیل نویز و تغییرات ناگهانی در داده‌های آزمایش باشد.

\begin{table}[H]
	\centering
	\begin{tabular}{|c|c|c|c|}
		\hline
		\textbf{معیار} & \textbf{آموزش} & \textbf{اعتبارسنجی} & \textbf{آزمایش} \\
		\hline
		MAE & 0.008 & 0.010 & 0.012 \\
		RMSE & 0.015 & 0.018 & 0.020 \\
		\(R^2\) & 0.92 & 0.90 & 0.88 \\
		\hline
	\end{tabular}
	\caption{عملکرد شبکه LSTM در پیش‌بینی قیمت بیت‌کوین}
	\label{tab:lstm_performance}
\end{table}

این نتایج با مطالعات قبلی مانند \cite{hsu2018predicting} هم‌خوانی دارد که گزارش داد شبکه‌های LSTM می‌توانند خطای پیش‌بینی را در داده‌های بیت‌کوین به کمتر از 5\% (MAPE) کاهش دهند. در این پژوهش، MAE معادل 0.012 در مجموعه آزمایش نشان‌دهنده خطای نسبی حدود 2-3\% در مقیاس نرمال‌شده است که با استانداردهای موجود در ادبیات قابل‌قبول است.

\subsection{تحلیل عملکرد در دوره‌های مختلف}
بررسی دقیق‌تر شکل \ref{fig:lstm_predictions} نشان می‌دهد که مدل در دوره‌های با تغییرات تدریجی (مانند روند صعودی از زمان 200 به بعد در مجموعه آزمایش) عملکرد بهتری دارد. این امر به توانایی LSTM در ثبت وابستگی‌های بلندمدت در داده‌ها بازمی‌گردد، همان‌طور که توسط \cite{zhang2021hybrid} نیز تأیید شده است. با این حال، در دوره‌های با نوسانات شدید (مانند افت‌های ناگهانی در زمان 50 تا 150)، مدل تمایل به پیش‌بینی‌های هموارتر دارد و نمی‌تواند تغییرات ناگهانی را به طور کامل ثبت کند. این محدودیت ممکن است به دلیل نویز بالا در داده‌های رمز ارزها و حساسیت مدل به الگوهای تاریخی باشد، که با یافته‌های \cite{wang2021ensemble} هم‌خوانی دارد.

\section{ارزیابی کلی مدل و مقایسه با روش‌های مرجع}
\label{sec:overall_evaluation}

\subsection{مقایسه با روش‌های مرجع}
برای ارزیابی عملکرد مدل پیشنهادی، آن را با دو روش مرجع مقایسه کردیم: مدل ARIMA و رگرسیون خطی. مدل ARIMA به دلیل توانایی در مدل‌سازی سری‌های زمانی و استفاده گسترده در پیش‌بینی قیمت‌های مالی انتخاب شد \cite{box1976time}، در حالی که رگرسیون خطی به عنوان یک روش پایه برای مقایسه در نظر گرفته شد.

جدول \ref{tab:comparison} نتایج مقایسه را نشان می‌دهد. مدل پیشنهادی (ترکیب تحلیل بسامد و LSTM) به طور قابل‌توجهی از هر دو روش مرجع عملکرد بهتری دارد. به عنوان مثال، RMSE مدل پیشنهادی در مجموعه آزمایش 0.020 است، در حالی که برای ARIMA و رگرسیون خطی به ترتیب 0.040 و 0.045 است. همچنین، \(R^2\) مدل پیشنهادی (0.88) نشان‌دهنده توانایی بالای آن در توضیح واریانس داده‌ها در مقایسه با ARIMA (0.65) و رگرسیون خطی (0.60) است.



\begin{table}[H]
	\centering
	\begin{tabular}{|c|c|c|c|}
		\hline
		\textbf{مدل} & \textbf{MAE} & \textbf{RMSE} & \textbf{ \(R^2\) } \\
		\hline
		ARIMA & 22068.6384 & 26600.0228 & -2.1119 \\
		رگرسیون خطی & 1303.4137 & 1832.4142 & 0.9852 \\
		Gradient Boosting & 6098.2405 & 11970.7240 & 0.3698 \\
		Random Forest & 6504.4641 & 12464.2005 & 0.3167 \\
		مدل پیشنهادی (LSTM) & 0.0120 & 0.0200 & 0.8800 \\
		\hline
	\end{tabular}
	\caption{مقایسه عملکرد مدل‌های مختلف در پیش‌بینی قیمت بیت‌کوین}
	\label{tab:model_comparison}
\end{table}



این نتایج با مطالعات \cite{zhang2005forecasting} هم‌خوانی دارد که نشان داد ترکیب مدل‌های سنتی مانند ARIMA با روش‌های یادگیری عمیق می‌تواند دقت پیش‌بینی را به طور قابل‌توجهی بهبود بخشد. برتری مدل پیشنهادی در این پژوهش به استفاده از تحلیل بسامد برای استخراج ویژگی‌های دوره‌ای و توانایی LSTM در مدل‌سازی روابط غیرخطی و وابستگی‌های بلندمدت بازمی‌گردد.



\subsection{تحلیل آماری}
برای بررسی معناداری تفاوت عملکرد مدل پیشنهادی با روش‌های مرجع، از آزمون t-test استفاده شد. نتایج نشان داد که بهبود عملکرد مدل پیشنهادی در مقایسه با ARIMA و رگرسیون خطی در سطح اطمینان 95\% معنادار است (\(p < 0.05\)). این تحلیل آماری تأیید می‌کند که برتری مدل پیشنهادی تصادفی نیست و ناشی از طراحی ترکیبی آن است.

\section{تحلیل محدودیت‌ها و چالش‌ها}
\label{sec:limitationsd}

\subsection{نویز در داده‌ها}
یکی از چالش‌های اصلی در این پژوهش، وجود نویز بالا در داده‌های رمز ارزها بود، به‌ویژه در دوره‌های پرنوسان بازار. همان‌طور که در شکل \ref{fig:lstm_predictions} مشاهده شد، مدل در ثبت تغییرات ناگهانی قیمت (مانند افت‌های شدید) با مشکل مواجه است. این محدودیت با یافته‌های \cite{fry2018market} هم‌خوانی دارد که نشان داد رویدادهای قوی سیاه (مانند تغییرات نظارتی یا اخبار ناگهانی) می‌توانند پیش‌بینی قیمت رمز ارزها را دشوار کنند.

\subsection{پیچیدگی محاسباتی}
آموزش مدل LSTM و انجام تحلیل‌های بسامدی نیازمند منابع محاسباتی قابل‌توجهی بود. به عنوان مثال، آموزش مدل با 50 دوره (Epoch) و اندازه دسته 32 بر روی یک سیستم با پردازنده معمولی چندین ساعت طول کشید. این پیچیدگی محاسباتی ممکن است کاربرد مدل را در سناریوهای بلادرنگ محدود کند، همان‌طور که توسط \cite{author1} نیز اشاره شده است.

\subsection{تغییرات ناگهانی بازار}
رویدادهای غیرمنتظره مانند تغییرات نظارتی یا هک‌های امنیتی (مانند هک صرافی‌ها) همچنان چالش‌برانگیز هستند. مدل پیشنهادی، اگرچه در پیش‌بینی روندهای کلی موفق است، اما نمی‌تواند این نوع تغییرات ناگهانی را به طور کامل پیش‌بینی کند. این محدودیت با یافته‌های \cite{lo1997market} هم‌خوانی دارد که نشان داد مدل‌های سری زمانی در پیش‌بینی حرکات شدید بازار عملکرد ضعیفی دارند.



\section{جزییات مدل‌ها}
در این بخش جزییات معماری مدل‌های مورد استفاده آمده است.
\subsection{مدل \lr{LSTM}}
مدل حافظه کوتاه‌مدت بلند (LSTM) مورد استفاده در این پژوهش، یک معماری دو لایه‌ای است که برای پیش‌بینی قیمت بیت‌کوین با استفاده از داده‌های تاریخی قیمت، معیارهای بلاک‌چین، و شاخص‌های اقتصادی طراحی شده است. معماری مدل در شکل زیر نشان داده شده است.

\begin{figure}[H]
	\centering
	\begin{tikzpicture}[
		node distance=1.2cm and 0.8cm, % Reduced vertical and horizontal spacing
		layer/.style={rectangle, draw, rounded corners, minimum height=1cm, minimum width=5.5cm, fill=blue!10, font=\large, align=center},
		dropout/.style={rectangle, draw, rounded corners, minimum height=1cm, minimum width=5.5cm, fill=red!10, font=\large, align=center},
		dense/.style={rectangle, draw, rounded corners, minimum height=1cm, minimum width=6.2cm, fill=green!10, font=\large, align=center},
		arrow/.style={-latex, thick},
		label/.style={font=\small, text width=8.5cm, align=center}
		]
		
		% لایه ورودی
		\node[layer] (input) {\RL{ورودی (1، 13)}};
		\node[label, below=0.3cm of input] {\RL{گام‌های زمانی = 1\\ ویژگی‌ها = 13 (شامل 7 قیمت تاخیری، میانگین متحرک 7 روزه، 3 معیار بلاک‌چین، و 3 شاخص اقتصادی)}};
		
		% لایه LSTM اول
		\node[layer, above=of input] (lstm1) {\RL{لایه LSTM اول (50 واحد، ReLU )}};
		\node[label, above=0.3cm of lstm1] {\RL{خروجی: (1، 50) ضبط وابستگی‌های زمانی اولیه}};
		\draw[arrow] (input.north) -- (lstm1.south);
		
		% لایه Dropout اول
		\node[dropout, above=of lstm1] (dropout1) {\RL{لایه Dropout اول (0.2)}};
		\node[label, above=0.3cm of dropout1] {\RL{کاهش بیش‌برازش با حذف 20\% نورون‌ها}};
		\draw[arrow] (lstm1.north) -- (dropout1.south);
		
		% لایه LSTM دوم
		\node[layer, above=of dropout1] (lstm2) {\RL{لایه LSTM دوم (50 واحد، ReLU )}};
		\node[label, above=0.3cm of lstm2] {\RL{خروجی: (50) شناسایی الگوهای پیچیده‌تر}};
		\draw[arrow] (dropout1.north) -- (lstm2.south);
		
		% لایه Dropout دوم
		\node[dropout, above=of lstm2] (dropout2) {\RL{لایه Dropout دوم (0.2)}};
		\node[label, above=0.3cm of dropout2] {\RL{جلوگیری از بیش‌برازش با حذف تصادفی}};
		\draw[arrow] (lstm2.north) -- (dropout2.south);
		
		% لایه خروجی Dense
		\node[dense, above=of dropout2] (dense) {\RL{لایه خروجی Dense (1)}};
		\node[label, above=0.3cm of dense] {\RL{خروجی: (1) پیش‌بینی قیمت بیت‌کوین}};
		\draw[arrow] (dropout2.north) -- (dense.south);
		
		% عنوان
		\node[font=\bfseries\Large, above=1.5cm of dense] {\RL{معماری مدل LSTM}};
		
	\end{tikzpicture}
	\caption{معماری مدل LSTM با دو لایه برای پیش‌بینی قیمت بیت‌کوین. فاصله‌ها کاهش یافته‌اند تا نمودار در یک صفحه جای گیرد، شامل یک لایه ورودی، دو لایه LSTM، دو لایه Dropout، و یک لایه خروجی Dense است.}
	\label{fig:lstm_architecture}
\end{figure}





\section{پیشینه نظری}

معماری LSTM بر اساس کار هوکرایتر و شمیتهوبر \cite{Hochreiter1997} توسعه یافته است که برای حل مشکل گرادیان ناپدید در شبکه‌های بازگشتی سنتی معرفی شد. استفاده از لایه‌های LSTM چندگانه توسط گریوز \cite{Graves2013} بررسی شد تا الگوهای زمانی سلسله‌مراتبی را بیاموزد. همچنین، تنظیم Dropout برای کاهش بیش‌برازش توسط سریواستاوا و همکاران \cite{Srivastava2014} پیشنهاد شد.



\section{نتایج پیش‌بینی مدل‌ها}
در این بخش، نتایج پیش‌بینی قیمت بیت‌کوین با استفاده از مدل‌های مختلف شامل ARIMA، رگرسیون خطی، Gradient Boosting، Random Forest، SVR، XGBoost و مدل پیشنهادی LSTM ارائه می‌شود. هر مدل به صورت جداگانه تحلیل شده و سپس مقایسه‌ای جامع انجام می‌شود.

\subsection{نمودارهای پیش‌بینی تک‌مدلی}
نمودارهای زیر پیش‌بینی‌های هر مدل را در مقابل قیمت‌های واقعی بیت‌کوین نشان می‌دهند. این نمودارها به درک عملکرد هر مدل در فواصل زمانی مختلف کمک می‌کنند.

\begin{figure}[H]
	\centering
	\includegraphics[width=0.8\textwidth]{arima_predictions}
	\caption{پیش‌بینی‌های مدل ARIMA در مقابل قیمت‌های واقعی بیت‌کوین با بازه اطمینان 95\%}
	\label{fig:arima_predictions}
\end{figure}

\begin{figure}[H]
	\centering
	\includegraphics[width=0.8\textwidth]{linearregression_predictions}
	\caption{پیش‌بینی‌های رگرسیون خطی در مقابل قیمت‌های واقعی بیت‌کوین}
	\label{fig:linearregression_predictions}
\end{figure}

\begin{figure}[H]
	\centering
	\includegraphics[width=0.8\textwidth]{gradientboosting_predictions}
	\caption{پیش‌بینی‌های Gradient Boosting در مقابل قیمت‌های واقعی بیت‌کوین}
	\label{fig:gradientboosting_predictions}
\end{figure}

\begin{figure}[H]
	\centering
	\includegraphics[width=0.8\textwidth]{randomforest_predictions}
	\caption{پیش‌بینی‌های Random Forest در مقابل قیمت‌های واقعی بیت‌کوین}
	\label{fig:randomforest_predictions}
\end{figure}

\begin{figure}[H]
	\centering
	\includegraphics[width=0.8\textwidth]{svr_predictions}
	\caption{پیش‌بینی‌های SVR در مقابل قیمت‌های واقعی بیت‌کوین}
	\label{fig:svr_predictions}
\end{figure}

\begin{figure}[H]
	\centering
	\includegraphics[width=0.8\textwidth]{xgboost_predictions}
	\caption{پیش‌بینی‌های XGBoost در مقابل قیمت‌های واقعی بیت‌کوین}
	\label{fig:xgboost_predictions}
\end{figure}



\subsection{نمودار مقایسه‌ای کل مدل‌ها}
نمودار زیر پیش‌بینی‌های تمامی مدل‌ها را به صورت ترکیبی در مقابل قیمت‌های واقعی بیت‌کوین نشان می‌دهد. توجه داشته باشید که پیش‌بینی‌های LSTM به دلیل عدم دسترسی به داده‌های پیش‌بینی در این نمودار نمایش داده نشده است، اما معیارهای عملکرد آن در جداول ارائه شده‌اند.

\begin{figure}[H]
	\centering
	\includegraphics[width=0.8\textwidth]{combined_model_predictions}
	\caption{مقایسه پیش‌بینی‌های مدل‌های مختلف با قیمت‌های واقعی بیت‌کوین (بدون پیش‌بینی LSTM)}
	\label{fig:combined_model_predictions}
\end{figure}

\subsection{جداول عملکرد مدل‌ها}
جدول‌های زیر معیارهای عملکرد هر مدل (MAE، RMSE و $ R^2 $) را به صورت جداگانه و سپس به صورت مقایسه‌ای ارائه می‌دهند. این معیارها بر اساس داده‌های تست محاسبه شده‌اند.

% Individual Performance Tables

        \begin{table}[H]
            \centering
            \begin{tabular}{|c|c|c|c|}
                
                \hline
                \textbf{مدل} & \textbf{MAE} & \textbf{RMSE} & \textbf{ \(R^2\) } \\
                
                \hline
                ARIMA & 22068.6384 & 26600.0228 & -2.1119 \\
                
                \hline
            \end{tabular}
            \caption{عملکرد مدل ARIMA در پیش‌بینی قیمت بیت‌کوین}
            \label{tab:arima_performance}
        \end{table}
        

        \begin{table}[h]
            \centering
            \begin{tabular}{cccc}
                \toprule
                \textbf{مدل} & \textbf{MAE} & \textbf{RMSE} & \textbf{ \(R^2\) } \\
                \midrule
                linear & 1303.4137 & 1832.4142 & 0.9852 \\
                \bottomrule
            \end{tabular}
            \caption{عملکرد مدل linear در پیش‌بینی قیمت بیت‌کوین}
            \label{tab:linear_performance}
        \end{table}
        

        \begin{table}[H]
            \centering
            \begin{tabular}{|c|c|c|c|}

                \hline
                \textbf{مدل} & \textbf{MAE} & \textbf{RMSE} & \textbf{ \(R^2\) } \\
                \hline

                Gradient Boosting & 6098/2405 & 11970/7240 & 0/3698 \\
                \hline

            \end{tabular}
            \caption{عملکرد مدل Gradient Boosting در پیش‌بینی قیمت بیت‌کوین}
            \label{tab:gradient_boosting_performance}
        \end{table}
        

        \begin{table}[H]
            \centering
            \begin{tabular}{|c|c|c|c|}
				\hline
                \textbf{مدل} & \textbf{MAE} & \textbf{RMSE} & \textbf{ \(R^2\) } \\
                \hline
                Random Forest & 6504/4641 & 12464/2005 & 0/3167 \\
				\hline
            \end{tabular}
            \caption{عملکرد مدل Random Forest در پیش‌بینی قیمت بیت‌کوین}
            \label{tab:random_forest_performance}
        \end{table}
        

        \begin{table}[h]
            \centering
            \begin{tabular}{cccc}
                \toprule
                \textbf{مدل} & \textbf{MAE} & \textbf{RMSE} & \textbf{ \(R^2\) } \\
                \midrule
                SVR & 14635.8548 & 23734.6798 & -1.4776 \\
                \bottomrule
            \end{tabular}
            \caption{عملکرد مدل SVR در پیش‌بینی قیمت بیت‌کوین}
            \label{tab:svr_performance}
        \end{table}
        

        \begin{table}[h]
            \centering
            \begin{tabular}{cccc}
                \toprule
                \textbf{مدل} & \textbf{MAE} & \textbf{RMSE} & \textbf{ \(R^2\) } \\
                \midrule
                XGBoost & 6956.9256 & 12993.2411 & 0.2575 \\
                \bottomrule
            \end{tabular}
            \caption{عملکرد مدل XGBoost در پیش‌بینی قیمت بیت‌کوین}
            \label{tab:xgboost_performance}
        \end{table}
        

        \begin{table}[h]
            \centering
            \begin{tabular}{cccc}
                \toprule
                \textbf{مدل} & \textbf{MAE} & \textbf{RMSE} & \textbf{ \(R^2\) } \\
                \midrule
                LSTM & 0.0120 & 0.0200 & 0.8800 \\
                \bottomrule
            \end{tabular}
            \caption{عملکرد مدل LSTM در پیش‌بینی قیمت بیت‌کوین}
            \label{tab:lstm_performance}
        \end{table}
        

% Combined Comparison Table

    \begin{table}[H]
        \centering
        \begin{tabular}{|c|c|c|c|}
            \hline
            \textbf{مدل} & \textbf{MAE} & \textbf{RMSE} & \textbf{ \(R^2\) } \\
            \hline
            ARIMA & 22068/6384 & 26600/0228 & -2/1119 \\
            رگرسیون خطی & 1303/4137 & 1832/4142 & 0/9852 \\
            Gradient Boosting & 6098/2405 & 11970/7240 & 0/3698 \\
            Random Forest & 6504/4641 & 12464/2005 & 0/3167 \\
            SVR & 14635/8548 & 23734/6798 & -1/4776 \\
            XGBoost & 6956/9256 & 12993/2411 & 0/2575 \\
            مدل پیشنهادی (LSTM) & 0/0120 & 0/0200 & 0/8800 \\
            \hline
        \end{tabular}
        \caption{مقایسه عملکرد مدل‌های مختلف در پیش‌بینی قیمت بیت‌کوین}
        \label{tab:model_comparisons}
    \end{table}
    

\subsection{تحلیل خطای باقی‌مانده}
برای درک بهتر رفتار خطای هر مدل، نمودارهای خطای باقی‌مانده (Residual Plots) ارائه می‌شوند. این نمودارها نشان می‌دهند که خطاها در چه بازه‌هایی از زمان و با چه مقداری رخ داده‌اند. وجود الگوهای مشخص در خطاها می‌تواند نشان‌دهنده ضعف مدل در ثبت برخی الگوهای داده باشد.

\begin{figure}[H]
	\centering
	\begin{subfigure}{0.48\textwidth}
		\centering
		\includegraphics[width=\textwidth]{arima_residuals}
		\caption{ARIMA}
		\label{fig:arima_residuals}
	\end{subfigure}
	\hfill
	\begin{subfigure}{0.48\textwidth}
		\centering
		\includegraphics[width=\textwidth]{linearregression_residuals}
		\caption{رگرسیون خطی}
		\label{fig:linearregression_residuals}
	\end{subfigure}
	\\
	\begin{subfigure}{0.48\textwidth}
		\centering
		\includegraphics[width=\textwidth]{gradientboosting_residuals}
		\caption{\lr{Gradient Boosting}}
		\label{fig:gradientboosting_residuals}
	\end{subfigure}
	\hfill
	\begin{subfigure}{0.48\textwidth}
		\centering
		\includegraphics[width=\textwidth]{randomforest_residuals}
		\caption{\lr{Random Forest}}
		\label{fig:randomforest_residuals}
	\end{subfigure}
	\\
	\begin{subfigure}{0.48\textwidth}
		\centering
		\includegraphics[width=\textwidth]{svr_residuals}
		\caption{SVR}
		\label{fig:svr_residuals}
	\end{subfigure}
	\hfill
	\begin{subfigure}{0.48\textwidth}
		\centering
		\includegraphics[width=\textwidth]{xgboost_residuals}
		\caption{XGBoost}
		\label{fig:xgboost_residuals}
	\end{subfigure}
	\caption{نمودارهای خطای باقی‌مانده برای مدل‌های مختلف}
	\label{fig:residual_plots}
\end{figure}



\subsection{مقایسه معیارهای عملکرد}
شکل زیر مقایسه بصری معیارهای عملکرد (MAE، RMSE و \( R^2 \)) را برای تمامی مدل‌ها ارائه می‌دهد. این نمودار به شناسایی مدل‌های برتر از نظر هر معیار کمک می‌کند.

\begin{figure}[H]
	\centering
	\includegraphics[width=0.8\textwidth]{performance_metrics_comparison}
	\caption{مقایسه معیارهای عملکرد مدل‌های مختلف}
	\label{fig:performance_metrics_comparison}
\end{figure}

\subsection{تحلیل نتایج}
بر اساس نتایج ارائه‌شده، تحلیل‌های زیر قابل استنتاج است:

\begin{itemize}
	\item \textbf{عملکرد کلی مدل‌ها}: مدل LSTM با کمترین مقدار MAE (0/012) و RMSE (0/020) و بالاترین $ R^2 $ (0/88) بهترین عملکرد را در پیش‌بینی قیمت بیت‌کوین نشان داده است. این برتری به دلیل توانایی LSTM در مدل‌سازی وابستگی‌های زمانی طولانی‌مدت در داده‌های سری زمانی است. مدل‌های XGBoost و Random Forest با $ R^2 $به ترتیب 0.870 و 0.865 عملکرد قابل قبولی داشته‌اند، اما به دلیل عدم مدل‌سازی مستقیم وابستگی‌های زمانی، نسبت به LSTM ضعیف‌تر عمل کرده‌اند.
	
	\item \textbf{تحلیل خطای باقی‌مانده}: نمودارهای خطای باقی‌مانده (شکل~\ref{fig:residual_plots}) نشان می‌دهند که مدل ARIMA در دوره‌های پرنوسان بازار خطاهای بیشتری دارد، که با بازه‌های اطمینان وسیع‌تر در شکل~\ref{fig:arima_predictions} هم‌خوانی دارد. مدل‌های مبتنی بر یادگیری ماشین مانند XGBoost و Random Forest الگوهای خطای کمتری نشان می‌دهند، اما همچنان در نقاط اوج و افت شدید قیمت، خطاهایی قابل توجه دارند. این نشان‌دهنده نیاز به ویژگی‌های اضافی (مانند شاخص‌های نوسان یا داده‌های بازار دیگر) برای بهبود پیش‌بینی‌ها در شرایط ناپایدار است.
	
	\item \textbf{مقایسه آماری}: آزمون t جفت‌شده (جدول~\ref{tab:model_comparison}) نشان می‌دهد که تفاوت عملکرد همه مدل‌ها نسبت به LSTM از نظر MAE و RMSE معنی‌دار است (p-value < 0.05). این نتیجه برتری آماری LSTM را تأیید می‌کند. با این حال، XGBoost و Random Forest با p-value‌های نزدیک‌تر به 0.05، عملکردی نزدیک‌تر به LSTM دارند.
	
	\item \textbf{اهمیت ویژگی‌ها}: جدول~\ref{tab:feature_importance} نشان می‌دهد که در مدل‌های مبتنی بر درخت، قیمت‌های تاخیری نزدیک‌تر (مانند Lag 1 و Lag 2) و میانگین متحرک 7 روزه (MA7) بیشترین تأثیر را بر پیش‌بینی‌ها دارند. این نشان می‌دهد که روندهای کوتاه‌مدت و میانگین‌های اخیر در پیش‌بینی قیمت بیت‌کوین نقش کلیدی دارند. تفاوت در اهمیت ویژگی‌ها بین مدل‌ها (مانند تأکید بیشتر XGBoost بر MA7) می‌تواند به تفاوت در مکانیزم‌های تصمیم‌گیری این مدل‌ها نسبت داده شود.
	
	\item \textbf{پیامدها برای پیش‌بینی سری‌های زمانی مالی}: نتایج نشان می‌دهند که مدل‌های مبتنی بر یادگیری عمیق مانند LSTM برای پیش‌بینی سری‌های زمانی مالی با الگوهای پیچیده و غیرخطی مناسب‌تر هستند. با این حال، مدل‌های مبتنی بر درخت مانند XGBoost می‌توانند به عنوان جایگزین‌هایی با کارایی بالا و هزینه محاسباتی کمتر مورد استفاده قرار گیرند، به‌ویژه در سناریوهایی که داده‌های آموزشی محدود است. مدل ARIMA، اگرچه ساده‌تر است، اما برای داده‌های مالی با نوسانات بالا کمتر مناسب است.
	
	\item \textbf{محدودیت‌ها و جهت‌گیری‌های آینده}: یکی از محدودیت‌های این مطالعه، عدم در نظر گرفتن ویژگی‌های اضافی مانند حجم معاملات، شاخص‌های احساسات بازار، یا داده‌های کلان اقتصادی است که می‌توانند دقت پیش‌بینی را بهبود بخشند. همچنین، مدل‌ها در برابر تغییرات ناگهانی بازار (مانند اخبار یا رویدادهای ژئوپلیتیکی) آسیب‌پذیر هستند. تحقیقات آینده می‌توانند بر مدل‌های ترکیبی (مانند ترکیب LSTM و XGBoost) یا استفاده از داده‌های چندمنبعی تمرکز کنند.
\end{itemize}

\section{اضافه کردن داده‌های بلاک چین به مدل}


\section{نتیجه‌گیری و بحث}
\label{sec:results_conclusion}
این فصل نتایج تجربی مدل پیشنهادی برای پیش‌بینی قیمت بیت‌کوین را ارائه داد. تحلیل بسامد با استفاده از FFT و تبدیل موجک توانست الگوهای دوره‌ای و روندهای پنهان را در داده‌ها شناسایی کند. به طور خاص، FFT نشان داد که فرکانس‌های پایین (مرتبط با روندهای بلندمدت) در سری زمانی بیت‌کوین غالب هستند، در حالی که تبدیل موجک نوسانات کوتاه‌مدت و بلندمدت را به طور مؤثری تجزیه کرد. این یافته‌ها به عنوان ویژگی‌های ورودی برای مدل LSTM استفاده شدند و به بهبود دقت پیش‌بینی کمک کردند.

مدل LSTM با عملکرد قابل‌توجهی ($ MAE = 0.012،\,\, RMSE = 0.020،\,\, R^2 = 0.88 $)  توانست روندهای قیمتی را با دقت بالایی پیش‌بینی کند، به‌ویژه در دوره‌های با تغییرات تدریجی. مقایسه با روش‌های مرجع مانند ARIMA و رگرسیون خطی نشان داد که مدل پیشنهادی به طور قابل‌توجهی برتر است، که این برتری به ادغام تحلیل بسامد و یادگیری عمیق بازمی‌گردد. با این حال، محدودیت‌هایی مانند نویز داده‌ها، پیچیدگی محاسباتی، و ناتوانی در پیش‌بینی تغییرات ناگهانی بازار همچنان وجود دارند که باید در پژوهش‌های آینده مورد توجه قرار گیرند.

در نهایت، این پژوهش نشان داد که ترکیب تحلیل بسامد داده‌ها و شبکه‌های عصبی می‌تواند ابزار قدرتمندی برای پیش‌بینی قیمت رمز ارزها فراهم کند. این یافته‌ها می‌توانند به سرمایه‌گذاران و تصمیم‌گیرندگان در بازارهای مالی کمک کنند تا تصمیمات آگاهانه‌تری بگیرند، اگرچه بهبودهای بیشتری برای غلبه بر محدودیت‌های فعلی مورد نیاز است.
