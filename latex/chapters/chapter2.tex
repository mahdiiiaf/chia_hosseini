\chapter{روش شناسی}


روش‌شناسی یکی از مهم‌ترین بخش‌های هر پژوهش علمی است که چارچوبی نظام‌مند برای طراحی، اجرا و ارزیابی مطالعه ارائه می‌دهد. در این پژوهش، هدف طراحی و تبیین مدلی برای پیش‌بینی قیمت رمز ارزها با استفاده از ترکیب روش‌های تحلیل بسامد داده‌ها، شبکه‌های عصبی و تکنیک‌های هوش مصنوعی در بازارهای مالی است. با توجه به پیچیدگی و پویایی بازار رمز ارزها، انتخاب روش‌شناسی مناسب برای تضمین اعتبار، پایایی و دقت نتایج امری ضروری است. 

این فصل به تشریح مراحل مختلف روش‌شناسی پژوهش پرداخته و شامل بخش‌هایی زیر است:
این فصل به بخش‌های زیر تقسیم شده است:
\begin{itemize}
	\item طراحی تحقیق
	\item جمع‌آوری داده‌ها
	\item پیش‌پردازش داده‌ها
	\item مهندسی ویژگی‌ها
	\item تحلیل بسامد داده‌ها
	\item طراحی شبکه‌های عصبی
	\item تکنیک‌های هوش مصنوعی
	\item ارزیابی مدل
	\item ملاحظات اخلاقی
	\item محدودیت‌ها و چالش‌ها
\end{itemize}




\section{طراحی تحقیق} 

این تحقیق هدف دارد مدلی برای پیش‌بینی قیمت رمز ارزها طراحی کند که مفاهیم تحلیل بسامد داده‌ها، شبکه‌های عصبی و تکنیک‌های هوش مصنوعی را ترکیب کند. تمرکز اصلی بر این است که چگونه رویکردهای مبتنی بر هوش مصنوعی، به ویژه روش‌های یادگیری عمیق، می‌توانند دقت و قابلیت اطمینان پیش‌بینی‌ها را در بازارهای پرنوسان رمز ارزها افزایش دهند. ترکیبی از تحلیل‌های سری زمانی، الگوریتم‌های یادگیری ماشین و تکنیک‌های پیشرفته هوش مصنوعی برای طراحی مدلی مقاوم استفاده خواهد شد. هدف اصلی این مطالعه این است که نشان دهد چگونه مدل‌های مبتنی بر هوش مصنوعی می‌توانند به سرمایه‌گذاران و مؤسسات مالی در پیش‌بینی قیمت رمز ارزها و اتخاذ تصمیمات سرمایه‌گذاری آگاهانه کمک کنند.

این پژوهش از رویکرد ترکیبی کمی و محاسباتی استفاده می‌کند که در آن داده‌های تاریخی رمز ارزها با استفاده از تکنیک‌های تحلیل بسامد، یادگیری عمیق و هوش مصنوعی تحلیل می‌شوند. رویکرد پیشنهادی شامل سه مؤلفه اصلی است:
\begin{enumerate}
	\item \textbf{تحلیل بسامد داده‌ها}: برای شناسایی الگوهای دوره‌ای و روندهای پنهان در سری‌های زمانی قیمت.
	\item \textbf{شبکه‌های عصبی}: برای مدل‌سازی روابط غیرخطی و پیچیده بین متغیرهای بازار.
	\item \textbf{هوش مصنوعی}: برای بهینه‌سازی مدل و تطبیق آن با شرایط متغیر بازار.
\end{enumerate}

مدل طراحی‌شده فرآیندی گام‌به‌گام را دنبال می‌کند که عبارت است از:

\begin{itemize} 
	\item \textbf{جمع‌آوری داده‌ها:} جمع‌آوری داده‌های تاریخی با کیفیت از رمز ارزها، از جمله بیت‌کوین، اتریوم و دیگر ارزهای دیجیتال. \item \textbf{پیش‌پردازش داده‌ها:} پاک‌سازی و نرمال‌سازی داده‌ها برای بهبود عملکرد مدل‌های یادگیری ماشین. \item \textbf{مهندسی ویژگی‌ها:} استخراج ویژگی‌های مرتبط با پیش‌بینی قیمت‌های آینده رمز ارزها. \item \textbf{تحلیل بسامد:} استفاده از تبدیلات فوریه برای تحلیل اجزای بسامدی داده‌های قیمت. \item \textbf{طراحی شبکه عصبی:} توسعه مدل‌های یادگیری عمیق برای پیش‌بینی حرکت قیمت رمز ارزها. \item \textbf{الگوریتم‌های هوش مصنوعی:} استفاده از تکنیک‌های هوش مصنوعی مانند یادگیری تقویتی برای بهینه‌سازی مدل. \item \textbf{ارزیابی مدل:} ارزیابی دقت، استحکام و قدرت پیش‌بینی مدل از طریق معیارهای مختلف عملکرد. 
\end{itemize}


\subsection{چارچوب مفهومی}
چارچوب مفهومی این پژوهش بر اساس ادغام تحلیل سری زمانی، یادگیری ماشین و هوش مصنوعی طراحی شده است. شکل \ref{fig:conceptual_framework} چارچوب پیشنهادی را نشان می‌دهد.

\begin{figure}[h]
	\centering
	\begin{tikzpicture}[node distance=2cm and 2cm, auto]
		% Nodes
		\node[cloud] (data) {جمع‌آوری داده‌ها};
		\node[block, below=of data] (preprocess) {پیش‌پردازش داده‌ها};
		\node[block, below=of preprocess] (freq) {تحلیل بسامد};
		\node[block, right=of freq] (features) {مهندسی ویژگی‌ها};
		\node[block, below=of freq] (nn) {شبکه‌های عصبی};
		\node[block, right=of nn] (ai) {تکنیک‌های هوش مصنوعی};
		\node[block, below=of nn] (eval) {ارزیابی مدل};
		\node[cloud, below=of eval] (output) {پیش‌بینی قیمت};
		
		% Arrows
		\draw[arrow] (data) -- (preprocess);
		\draw[arrow] (preprocess) -- (freq);
		\draw[arrow] (freq) -- (features);
		\draw[arrow] (freq) -- (nn);
		\draw[arrow] (features) -- (nn);
		\draw[arrow] (nn) -- (ai);
		\draw[arrow] (ai) -- (eval);
		\draw[arrow] (nn) -- (eval);
		\draw[arrow] (eval) -- (output);
		
		% Background
		\begin{scope}[on background layer]
			\node[fill=gray!10, rounded corners, fit=(data) (preprocess) (freq) (features) (nn) (ai) (eval) (output), inner sep=0.5cm] {};
		\end{scope}
	\end{tikzpicture}
	\caption{چارچوب مفهومی مدل پیش‌بینی قیمت رمز ارزها}
	\label{fig:conceptual_framework}
\end{figure}

\subsection{نوع پژوهش}
این پژوهش از نوع کاربردی و توسعه‌ای است، زیرا هدف آن طراحی و اعتبارسنجی مدلی عملی برای پیش‌بینی قیمت رمز ارزها در بازارهای مالی است. رویکرد پژوهش ترکیبی از روش‌های اکتشافی (برای شناسایی ویژگی‌ها) و آزمایشی (برای ارزیابی مدل) است.

\subsection{فرآیند پژوهش}
فرآیند پژوهش به‌صورت گام‌به‌گام طراحی شده است:
\begin{enumerate}
	\item جمع‌آوری داده‌های تاریخی رمز ارزها.
	\item پیش‌پردازش داده‌ها برای حذف نویز و نرمال‌سازی.
	\item استخراج ویژگی‌های کلیدی با استفاده از تحلیل بسامد و روش‌های آماری.
	\item طراحی و آموزش مدل‌های شبکه عصبی.
	\item بهینه‌سازی مدل با استفاده از تکنیک‌های هوش مصنوعی.
	\item ارزیابی مدل با معیارهای عملکرد.
\end{enumerate}

\section{جمع‌آوری داده‌ها}
\label{sec:data_collection}

\subsection{منابع داده}
داده‌های مورد استفاده در این پژوهش شامل داده‌های تاریخی قیمت رمز ارزها (مانند بیت‌کوین، اتریوم و ریپل) و معیارهای مرتبط بازار است. منابع اصلی داده‌ها عبارتند از:
\begin{itemize}
	\item \textbf{صرافی‌های رمز ارز}: مانند Binance، Coinbase و Kraken.
	\item \textbf{پلتفرم‌های داده مالی}: مانند CoinMarketCap، CoinGecko و \lr{Yahoo Finance}.
	\item \textbf{داده‌های اجتماعی}: استخراج احساسات بازار از توییتر و ردیت با استفاده از APIهای مربوطه.
\end{itemize}

\subsection{نوع داده‌ها}
داده‌ها در دو دسته اصلی جمع‌آوری می‌شوند:
\begin{itemize}
	\item \textbf{داده‌های با فرکانس بالا}: شامل داده‌های تیک‌به‌تیک و دقیقه‌ای برای ثبت تغییرات کوتاه‌مدت.
	\item \textbf{داده‌های با فرکانس پایین}: شامل داده‌های ساعتی، روزانه و هفتگی برای تحلیل روندهای بلندمدت.
\end{itemize}

\subsection{دوره زمانی}
داده‌ها برای دوره زمانی ژانویه ۲۰۱۸ تا دسامبر ۲۰۲۴ جمع‌آوری می‌شوند تا شامل چرخه‌های مختلف بازار (صعودی، نزولی و تثبیت) باشند. این بازه زمانی به مدل امکان می‌دهد تا الگوهای متنوعی را یاد بگیرد.

\subsection{ابزارهای جمع‌آوری}
برای جمع‌آوری داده‌ها از ابزارهای زیر استفاده می‌شود:
\begin{itemize}
	\item \textbf{Python Libraries}: کتابخانه‌هایی مانند \texttt{ccxt}، \texttt{yfinance} و \texttt{tweepy} برای استخراج داده‌ها.
	\item \textbf{APIها}: APIهای صرافی‌ها و پلتفرم‌های مالی برای دسترسی به داده‌های بلادرنگ.
	\item \textbf{پایگاه‌های داده‌ها}: \footnote{Database}ذخیره‌سازی داده‌ها در پایگاه‌های داده‌های SQL (مانند PostgreSQL) برای مدیریت کارآمد.
\end{itemize}


\subsection{چالش‌های جمع‌آوری داده}
چالش‌های اصلی شامل موارد زیر است:
\begin{itemize}
	\item \textbf{کیفیت داده}: داده‌های ناقص یا نادرست از برخی منابع.
	\item \textbf{هزینه دسترسی}: برخی APIها نیازمند اشتراک پولی هستند.
	\item \textbf{حجم داده}: مدیریت داده‌های با فرکانس بالا نیازمند زیرساخت محاسباتی قوی است.
\end{itemize}

\section{پیش‌پردازش داده‌ها}
\label{sec:data_preprocessing}

\subsection{پاک‌سازی داده‌ها}
داده‌های خام اغلب شامل نویز، مقادیر گمشده و ناهنجاری‌ها هستند. مراحل پاک‌سازی شامل:
\begin{itemize}
	\item \textbf{حذف مقادیر گمشده}: پر کردن مقادیر گمشده با روش‌هایی مانند میانگین متحرک یا درون‌یابی خطی.
	\item \textbf{حذف ناهنجاری‌ها}: شناسایی و حذف مقادیر پرت با استفاده از روش‌های آماری مانند Z-score.
	\item \textbf{یکپارچه‌سازی داده‌ها}: همگام‌سازی داده‌ها از منابع مختلف برای اطمینان از هماهنگی زمانی.
\end{itemize}

\subsection{نرمال‌سازی}
برای بهبود عملکرد مدل‌های یادگیری ماشین، داده‌ها نرمال‌سازی می‌شوند:
\begin{itemize}
	\item \textbf{Min-Max Scaling}: مقیاس‌بندی داده‌ها به بازه [0,1].
	\item \textbf{Standardization}: تبدیل داده‌ها به توزیع با میانگین صفر و انحراف معیار یک.
\end{itemize}

\subsection{تقسیم داده‌ها}
داده‌ها به سه مجموعه تقسیم می‌شوند:
\begin{itemize}
	\item \textbf{آموزش} (70\%): برای آموزش مدل.
	\item \textbf{اعتبارسنجی} (15\%): برای تنظیم پارامترهای مدل.
	\item \textbf{آزمایش} (15\%): برای ارزیابی نهایی مدل.
\end{itemize}

\subsection{ابزارهای پیش‌پردازش}
ابزارهای مورد استفاده شامل:
\begin{itemize}
	\item \textbf{Pandas}: برای مدیریت و پاک‌سازی داده‌ها.
	\item \textbf{NumPy}: برای عملیات ریاضی و نرمال‌سازی.
	\item \textbf{Scikit-learn}: برای پیش‌پردازش و تقسیم داده‌ها.
\end{itemize}

\section{مهندسی ویژگی‌ها}
\label{sec:feature_engineering}

\subsection{ویژگی‌های اولیه}


\section{تحلیل بسامد داده‌ها}
\label{sec:frequency_analysis}

\subsection{تبدیل فوریه}
برای تحلیل بسامد داده‌ها از تبدیل فوریه سریع (FFT) استفاده می‌شود:
\begin{equation}
	X(k) = \sum_{n=0}^{N-1} x(n) e^{-j2\pi kn/N}
\end{equation}
جایی که \(X(k)\) طیف فرکانسی، \(x(n)\) سری زمانی و \(N\) تعداد نمونه‌ها است.

\subsection{تبدیل موجک}
تبدیل موجک برای تجزیه داده‌ها به مقیاس‌های مختلف استفاده می‌شود:
\begin{equation}
	W(s,\tau) = \int_{-\infty}^{\infty} x(t) \psi^*\left(\frac{t-\tau}{s}\right) dt
\end{equation}
جایی که \(\psi\) تابع موجک، \(s\) مقیاس و \(\tau\) جابجایی زمانی است.

\section{طراحی شبکه‌های عصبی}
\label{sec:neural_networks}

\subsection{شبکه‌های عصبی پیش‌خور (FNN)}
FNN‌ها برای مدل‌سازی روابط غیرخطی استفاده می‌شوند. ساختار شامل:
\begin{itemize}
	\item لایه ورودی: ویژگی‌های استخراج‌شده.
	\item لایه‌های مخفی: با فعال‌سازی ReLU.
	\item لایه خروجی: پیش‌بینی قیمت.
\end{itemize}

\subsection{شبکه‌های عصبی بازگشتی (RNN)}
RNN‌ها برای داده‌های توالی‌ای طراحی شده‌اند:
\begin{equation}
	h_t = \tanh(W_{hh}h_{t-1} + W_{xh}x_t + b_h)
\end{equation}

\subsection{شبکه‌های LSTM}
LSTM‌ها برای ثبت وابستگی‌های بلندمدت استفاده می‌شوند:
\begin{equation}
	f_t = \sigma(W_f \cdot [h_{t-1}, x_t] + b_f)
\end{equation}

\section{تکنیک‌های هوش مصنوعی}
\label{sec:ai_techniques}

\subsection{یادگیری تقویتی}
یادگیری تقویتی برای بهینه‌سازی استراتژی‌های معاملاتی استفاده می‌شود:
\begin{equation}
	Q(s,a) \leftarrow Q(s,a) + \alpha [r + \gamma \max Q(s',a') - Q(s,a)]
\end{equation}

\subsection{الگوریتم‌های ژنتیک}
الگوریتم‌های ژنتیک برای بهینه‌سازی پارامترهای مدل استفاده می‌شوند.

\section{ارزیابی مدل}
\label{sec:model_evaluation}

\subsection{معیارهای عملکرد}
معیارهای ارزیابی شامل:
\begin{itemize}
	\item میانگین خطای مطلق (MAE):
	\begin{equation}
		\text{MAE} = \frac{1}{n} \sum_{i=1}^n |y_i - \hat{y}_i|
	\end{equation}
	\item ریشه میانگین مربعات خطا (RMSE):
	\begin{equation}
		\text{RMSE} = \sqrt{\frac{1}{n} \sum_{i=1}^n (y_i - \hat{y}_i)^2}
	\end{equation}
	\item ضریب تعیین (\(R^2\)):
	\begin{equation}
		R^2 = 1 - \frac{\sum (y_i - \hat{y}_i)^2}{\sum (y_i - \bar{y})^2}
	\end{equation}
\end{itemize}

\subsection{اعتبارسنجی متقاطع}
از روش \lr{K-fold cross-validation} برای ارزیابی تعمیم‌پذیری مدل استفاده می‌شود.

\section{ملاحظات اخلاقی}
\label{sec:ethical_considerations}
این پژوهش به اصول اخلاقی پایبند است:
\begin{itemize}
	\item شفافیت در استفاده از داده‌ها.
	\item اجتناب از سوءاستفاده از مدل برای دستکاری بازار.
	\item رعایت حریم خصوصی در داده‌های اجتماعی.
\end{itemize}

\section{محدودیت‌ها و چالش‌ها}
\label{sec:limitations}
چالش‌های اصلی شامل:
\begin{itemize}
	\item نویز در داده‌های با فرکانس بالا.
	\item پیچیدگی محاسباتی مدل‌های یادگیری عمیق.
	\item تغییرات سریع بازار رمز ارزها.
\end{itemize}

\section{نتیجه‌گیری}
این فصل روش‌شناسی جامع و دقیقی برای طراحی و اعتبارسنجی مدل پیش‌بینی قیمت رمز ارزها ارائه داد. با ادغام تحلیل بسامد، شبکه‌های عصبی و هوش مصنوعی، این پژوهش گامی نوآورانه در پیش‌بینی بازارهای مالی برمی‌دارد.
