\chapter{مرور ادبیات}
\section{مقدمه}
رمز ارزهااز زمان معرفی بیت‌کوین در سال ۲۰۰۹ به‌عنوان یکی از نوآوری‌های پررنگ و تغییر پارادایم و چارچوب در حوزه مالی مدرن ظاهر شده‌اند. بسیاری از مفاهیم مدرن مالی پس از ظهور این دارایی به وجود آمدند یا حداقل مرکز توجه قرار گرفتند. برخی از این مفاهیم عبارتند مالی غیر متمرکز \footnote{\lr{Decentralized Finance (DeFi)}}، بلاک چین \footnote{\lr{Blockchain}}، فین تک \footnote{\lr{Financial Technology (FinTech)}}. طبیعت غیرمرکزی این گونه دارایی‌های مالی، همراه با پذیرش سریع آنها در بین مردم و نوسانات بالا، آنها را به یک دسته دارایی جذاب و همچنین چالش‌برانگیز برای سرمایه‌گذاران، معامله‌گران و پژوهشگران و محققین تبدیل کرده است. پیش‌بینی دقیق قیمت رمز ارزها به دلیل تعامل پیچیده عوامل مختلفی که بازار را تحت تأثیر قرار می‌دهند، شامل روند اقتصادی کلان، توسعه فناوری، تغییرات نظارتی و رفتار‌های احساسی کنشگران بازار همچنان یک چالش اساسی است.

پیشرفت‌های اخیر در یادگیری ماشین (ML)، یادگیری عمیق (DL) و هوش مصنوعی (AI) راه‌های جدیدی را برای مقابله با این چالش ایجاد کرده است. این فناوری‌ها به ما اجازه شناسایی و امکان تشخیص مواردی چون الگوهای پیچیده، انحراف معیار و پیش‌بینی را می‌دهند که قبلاً با استفاده از مدل‌های اقتصادسنجی سنتی قابل دسترس نبودند. علاوه بر این، تحلیل دامنه بسامد نقش تکمیلی را در افزایش دقت پیش‌بینی ایفا می‌کند که با شناسایی الگوهای دوره‌ای و رفتارهای روندی در داده‌های سری زمانی امکان‌پذیر می‌شود.


در ادامه ما در اینجا به بررسی تاریخچه پیش بینی، تئوری‌ها و روش‌های سنتی و در ادامه روش‌های جدید استفاده در ادبیات شامل ترکیب روش‌های فرکانس داده، شبکه‌های عصبی و تکنیک‌های هوش مصنوعی در پیش‌بینی قیمت دارایی‌های مالی و ارزهای دیجیتال خواهیم پرداخت. ابتدا مطالعات مربوط به روش‌های فرکانس داده در بازارهای مالی را مرور می‌کنیم و سپس به بررسی شبکه‌های عصبی، هوش مصنوعی و مدل‌های یادگیری ماشین مورد استفاده در پیش‌بینی ارزهای دیجیتال خواهیم پرداخت. در نهایت، نقاط قوت و ضعف این رویکردها را تحلیل کرده و جهت‌گیری‌های آینده تحقیق را مورد بررسی قرار می‌دهیم.

\section{تاریخچه پیش بینی قیمت دارایی‌های مالی}
پیش‌بینی قیمت ابزارهای مالی، از جمله سهام، اوراق قرضه، کالاها و نرخ‌های ارز، همواره موضوعی جالب و مهم در دنیای آکادمیک و صنعت مالی بوده است. پیش‌بینی دقیق قیمت‌ها برای اتخاذ تصمیمات مؤثر در زمینه‌های معاملاتی، مدیریت ریسک و استراتژی‌های سرمایه‌گذاری ضروری است. در طول سال‌ها، مدل‌ها و تکنیک‌های مختلفی برای پیش‌بینی قیمت‌ها پیشنهاد و آزمایش شده است. این مدل‌ها به طور چشمگیری تکامل یافته‌اند و این تکامل به پیشرفت‌های نظریه‌های آماری، تکنیک‌های محاسباتی و در دسترس بودن داده‌ها برمی‌گردد. این بخش به بررسی تاریخچه پیش‌بینی قیمت ابزارهای مالی پرداخته و تکامل مدل‌ها و روش‌شناسی‌ها از رویکردهای کلاسیک تا مدل‌های مدرن یادگیری ماشین و هوش مصنوعی را مرور می‌کند.

\subsubsection{فرضیه بازار کارا (\lr{EMH})}
یکی از نخستین و تأثیرگذارترین نظریه‌ها در اقتصاد مالی، فرضیه بازار کارا\footnote{\lr{Efficient Market Hypothesis}} است که توسط اوژن فاما در دهه ۱۹۶۰ معرفی شد \cite{fama1965efficient}. این فرضیه بیان می‌کند که بازارهای مالی «کارا» هستند، به این معنا که قیمت‌ها تمامی اطلاعات موجود در هر زمان خاص را به طور کامل منعکس می‌کنند. طبق این نظریه، قیمت‌های سهام پیروی از یک فرآیند تصادفی دارند، به این معنی که حرکت‌های قیمت در آینده قابل پیش‌بینی نیستند و نمی‌توان از داده‌های گذشته برای پیش‌بینی استفاده کرد. این نظریه نشان می‌دهد که دستیابی به بازدهی بیشتر از میانگین بازار به‌طور مداوم از طریق استفاده از داده‌های تاریخی غیرممکن است.


فرضیه بازار کارا منجر به توسعه مدل‌هایی شد که فرض می‌کردند حرکت‌های قیمت تصادفی بوده و داده‌های گذشته نمی‌تواند بینش‌های مفیدی برای پیش‌بینی قیمت‌ها فراهم کند. به همین دلیل، بسیاری از مدل‌های اولیه استفاده‌شده در پیش‌بینی قیمت‌های مالی بر پایه فرآیندهای تصادفی و پیاده‌سازی قدم زن تصادفی \footnote{\lr{Random Walk}}بودند. در حالی که فرضیه بازار کارا در زمان خود انقلابی بود، به دلیل ناتوانی در توضیح پدیده‌های بازار نظیر حباب‌ها، سقوط‌ها و رفتار سرمایه‌گذاران که همیشه با فرض کارایی بازار هم‌خوانی ندارند، مورد انتقاد قرار گرفت.


\subsubsection{نقدهای تجربی به فرضیه بازارا کارا}
تحقیقات تجربی بسیاری به ارزیابی اعتبار \lr{EMH} پرداخته‌اند و نمونه‌هایی از انحرافات بازار از مفروضات \lr{EMH} ارائه داده‌اند. \cite{lo1997market} یکی از برجسته‌ترین محققان در این زمینه است که با ارزیابی تغییرات قیمت در بازارهای مختلف به این نتیجه رسید که \lr{EMH} قادر به توضیح تمامی نوسانات و بحران‌های بازار نیست. به‌ویژه، بررسی‌های او نشان داد که برخی از بازارها، نظیر بازار سهام، به‌طور مداوم از ارزش‌های بنیادی خود انحراف دارند.

مطالعه‌ی \cite{festinger1957theory} در زمینه ناهنجاری‌های شناختی\footnote{\lr{cognitive dissonance}} نیز نشان داد که سرمایه‌گذاران در بسیاری از مواقع به دنبال تأیید اطلاعاتی هستند که با باورهای قبلی آن‌ها همخوانی دارد و این می‌تواند منجر به تصمیمات نادرست و انحراف قیمت‌ها از ارزش‌های منطقی شود.

\subsubsection{توسعه‌های مدل‌های جایگزین و نقدهای مدرن به فرضیه بازار کارا }

\subsubsection{نظریه بازارهای غیرکارا}

از دهه ۱۹۹۰ به بعد، مفهوم بازارهای غیرکارا \footnote{\lr{Inefficient Markets}} به یکی از نقدهای برجسته به فرضییه کارایی بازار تبدیل شد. \cite{malkiel2003efficient} در بررسی‌های خود به این نتیجه رسید که بسیاری از بازارها به‌ویژه در دوران‌هایی که نوسانات شدید را تجربه می‌کنند، نمی‌توانند به‌طور کامل٫ کارا و سریع به اطلاعات جدید واکنش نشان دهند. در واقع، به‌دلیل وجود موانع اطلاعاتی، اصطحکاک‌ها\footnote{Frictions}٫ هنجارهای مالی نادرست و رفتارهای غیرمنطقی سرمایه‌گذاران، ممکن است همیشه بازارهایی با رفتار غیرکارا وجود داشته باشند.

\cite{fama1991efficient} در نقدهای خود بیان کردند که مشکلات اطلاعاتی و شرایط نامساعد اقتصادی می‌تواند بر کارایی بازار تأثیر بگذارد و باعث کند شدن فرآیند انطباق قیمت‌ها با اطلاعات جدید شود. بنابراین، به گفته‌ی او، در برخی مواقع، به‌ویژه در بازارهای نوسانی، اثربخشی \lr{EMH} می‌تواند تحت تأثیر این عوامل قرار گیرد.

\subsubsection{تحقیقات جدید و گسترش فرضیه کارایی بازار}
برخی از پژوهشگران سعی کرده‌اند تا فرضیه کارایی بازار را با استفاده از مدل‌های جدیدتر بسط و گسترش دهند. به‌عنوان‌مثال، \cite{fama2014five} به روز رسانی‌های اخیر در این زمینه اشاره کرد و مدل‌های اصلاح‌شده‌ای را پیشنهاد داد که در آن‌ها بر نقش اطلاعات نهفته و تأثیر رفتارهای گروهی تأکید شده است.

مطالعه‌ی \cite{poterba1992stock} نشان داد که برخی از نوسانات سهام که در بلندمدت مشاهده می‌شود، به‌طور غیرمنطقی موجب تغییرات عمده در قیمت‌ها می‌شود و این تغییرات در واقع تحت تأثیر اطلاعات غیررسمی، شایعات و تصمیمات اشتباه تحلیل‌گران است.

\subsubsection{نظرات حامیان فرضیه کارایی بازار}

با وجود نقدهای متعددی که به فرضیه کارایی بازار وارد شده، حامیان این نظریه همچنان به مزایای آن اشاره دارند. \cite{malkiel2003efficient} همچنان بر این باور است که در بلندمدت، بازارهای کارا به‌طور مؤثری قیمت‌ها را براساس اطلاعات موجود تنظیم می‌کنند و با استفاده از داده‌های تاریخی نمی‌توان پیش‌بینی‌های بهتری نسبت به قدم زن تصادفی در بازارهای کارا انجام داد.

\cite{fama1991efficient} نیز تأکید می‌کند که بسیاری از انحرافات در بازارهای مالی ناشی از رفتارهای کوتاه‌مدت و واکنش‌های هیجانی است که نمی‌توان آن‌ها را به عنوان دلیل عدم کارایی بازارها در نظر گرفت.

\subsubsection{نتیجه گیری نقد‌ها به کارایی بازار}
در حالی که فرضیه کارایی بازار به عنوان یکی از مفاهیم و فرضیه‌های کلیدی در اقتصاد و مالی شناخته می‌شود، یافته‌ها نشان می‌دهد که این نظریه قادر به توضیح تمامی ویژگی‌های بازارهای مالی به‌ویژه در مواجهه با بحران‌ها و نوسانات غیرمنتظره نیست. در نتیجه، نظریه مورد پذیرش عمده متخصصین مالی این است که بازارها در برخی موارد کارا هستند و در برخی دیگر، به‌ویژه در شرایط خاص، ممکن است عملکرد غیرکارا از خود نشان دهند.
\section{روش‌ها و مدل‌های مورد استفاده برای پیش بینی}
\subsection{مدل‌های \lr{ARIMA}}
مدل \lr{ARIMA}\footnote{\lr{Autoregressive Integrated Moving Average}} که توسط جورج باکس و گویلیم جنکینز در ۱۹۷۰ معرفی شد \cite{box1976time}، یکی از اولین روش‌های آماری برای پیش‌بینی سری‌های زمانی مالی بود. از آن زمان به بعد مدل‌های \lr{ARIMA} به‌طور گسترده‌ای برای پیش‌بینی قیمت سهام و نرخ‌های ارز٫ به‌ویژه در زمینه پیش‌بینی‌های کوتاه‌مدت٫ استفاده می‌شوند. 

این مدل بر این فرضی استوار است که مقادیر آینده یک سری زمانی را می‌توان با ترکیب خطی از مقادیر گذشته و خطاهای آن پیش‌بینی کرد.

مدل \lr{ARIMA} از سه مؤلفه تشکیل شده است:
\begin{itemize}
	\item مؤلفه خودرگرسیو (\lr{AR}): رگرسیون خطی از مقدار فعلی بر روی مقادیر گذشته.
	\item مؤلفه یکپارچگی (\lr{I}): تفاوت‌گیری سری برای ایستا کردن آن (یعنی حذف روندها).
	\item مؤلفه میانگین متحرک (\lr{MA}): ترکیب خطی از خطاهای گذشته.
\end{itemize}

مدل \lr{ARIMA} فرض می‌کند که سری‌های زمانی مالی ایستا هستند، به این معنا که ویژگی‌های آماری آن‌ها در طول زمان تغییر نمی‌کند. در حالی که مدل‌های \lr{ARIMA} در برخی موارد در پیش‌بینی قیمت ابزارهای مالی موفق بوده‌اند، محدودیت‌هایی نیز دارند. به عنوان مثال، این مدل‌ها نمی‌توانند الگوهای پیچیده و روابط غیرخطی که معمولاً در بازارهای مالی وجود دارند را مدل کنند.
\subsubsection{کاربردهای \lr{ARIMA} در پیش‌بینی قیمت‌ها}
این مدل در ادبیات کاربرد‌های بسیاری متنوعی پیدا کرده است که از مهم ترین آنها می‌توان به موارد زیر اشاره کرد.

\textbf{پیش‌بینی قیمت سهام}


یکی از اولین کاربردهای مدل \lr{ARIMA} در پیش‌بینی قیمت سهام بود. \cite{box1976time} و \cite{weigend1991predicting} از \lr{ARIMA} برای پیش‌بینی قیمت سهام در بازه‌های زمانی کوتاه‌مدت استفاده کردند. در این مطالعات، پیش‌بینی قیمت‌ها با استفاده از داده‌های تاریخی انجام شد و نشان داده شد که \lr{ARIMA} قادر است تغییرات قیمت سهام را در مقیاس زمانی روزانه و هفتگی پیش‌بینی کند. با این حال، این مدل‌ها اغلب در بلندمدت کارایی کمتری دارند و نوسانات غیرمنتظره بازار را به خوبی نمی‌توانند مدل کنند.

\textbf{پیش‌بینی نرخ ارز}

در مقالات متعددی، \lr{ARIMA} برای پیش‌بینی نرخ ارز نیز به‌کار گرفته شده است. \cite{tsay2005analysis} و \cite{\} نشان دادند که مدل‌های \lr{ARIMA} به ویژه در پیش‌بینی نرخ ارزهای کشورهای در حال توسعه، عملکرد خوبی دارند. این مطالعات نشان دادند که با استفاده از داده‌های تاریخی نرخ ارز، مدل‌های \lr{ARIMA} قادرند  تا تغییرات نرخ ارز را در بازه‌های زمانی کوتاه‌مدت با دقت نسبتاً خوبی پیش‌بینی کنند.

\textbf{پیش‌بینی قیمت نفت}

مدل‌های \lr{ARIMA} در پیش‌بینی قیمت نفت نیز مورد استفاده قرار گرفته‌اند. \cite{zhang2008forecasting} مدل‌های \lr{ARIMA} را برای پیش‌بینی قیمت نفت با استفاده از داده‌های ماهانه استفاده کردند. نتایج آن‌ها نشان داد که این مدل‌ها می‌توانند روندهای کلی بازار نفت را پیش‌بینی کنند، اما در پیش‌بینی‌های دقیق‌تر و در فواصل زمانی کوتاه‌تر با چالش‌هایی مواجه هستند.

\subsection{نقدها و محدودیت‌های مدل \lr{ARIMA}}

\subsubsection{محدودیت‌ در مدل‌سازی روابط غیرخطی}
یکی از اصلی‌ترین محدودیت‌های مدل \lr{ARIMA} این است که این مدل‌ها بر اساس فرضیه خطی بودن داده‌ها عمل می‌کنند. \cite{lo1997market} به این نکته اشاره کردند که بسیاری از بازارهای مالی روابط غیرخطی دارند و مدل \lr{ARIMA} قادر به مدل‌سازی چنین روابطی نیست. در این خصوص، مطالعاتی همچون \cite{anderson1998forecasting} نشان دادند که مدل‌های \lr{ARIMA} در مواجهه با نوسانات غیرمنتظره و تغییرات شدید در قیمت‌ها عملکرد ضعیفی دارند.

\subsubsection{عدم توانایی در پیش‌بینی تغییرات ناگهانی}
یکی دیگر از مشکلات مدل \lr{ARIMA} این است که این مدل قادر به پیش‌بینی تغییرات ناگهانی و نوسانات غیرمنتظره در بازار نیست. \cite{diebold2001forecasting} نشان داد که \lr{ARIMA} نمی‌تواند بحران‌های مالی و حباب‌های بازار را پیش‌بینی کند. در حالی که این مدل برای پیش‌بینی روندهای کلی بازار مناسب است، نمی‌تواند برای پیش‌بینی تغییرات ناگهانی و رفتارهای غیرخطی مؤثر باشد.

\subsubsection{چالش‌های مربوط به ایستایی داده‌ها}
مدل \lr{ARIMA}  بر فرض ایستا بودن داده‌ها دارد. به عبارت دیگر، داده‌های ورودی باید در طول زمان ثابت باشند ، که در بسیاری از موارد این شرط برآورده نمی‌شود. ثابت بودن و ایستایی در اینجا به معنای ثابت بودن تکانه اول و دوم و یا ثابت بودن میانگین و انحراف معیار آن است. \cite{hamilton1994time} بیان کردند که برای استفاده مؤثر از مدل \lr{ARIMA} باید داده‌ها را پیش از استفاده به فرم ایستا تبدیل کرد، که این فرآیند ممکن است باعث از دست دادن اطلاعات ارزشمند شود.

\subsection{پیشرفت‌ها و گسترش‌های مدل \lr{ARIMA}}

\subsubsection{مدل‌های ترکیبی \lr{ARIMA}}
برای غلبه بر محدودیت‌های مدل \lr{ARIMA}، محققان مدل‌های ترکیبی را پیشنهاد داده‌اند که از \lr{ARIMA} به همراه دیگر روش‌ها مانند شبکه‌های عصبی و مدل‌های یادگیری ماشین استفاده می‌کنند. \cite{armstrong2001forecasting} و \cite{zhang2005forecasting} مدل‌های ترکیبی \lr{ARIMA} و شبکه‌های عصبی را برای پیش‌بینی قیمت‌ها در بازارهای مالی پیشنهاد کردند و نشان دادند که این ترکیب‌ها دقت پیش‌بینی را به طور قابل توجهی افزایش می‌دهند.

\subsubsection{مدل‌های \lr{ARIMA-GARCH}}

برای پیش‌بینی نوسانات قیمت و مدل‌سازی وابستگی‌های بلندمدت، مدل‌های \lr{ARIMA} با مدل‌های \lr{GARCH}\footnote{\lr{Generalized Autoregressive Conditional Heteroskedasticity}} ترکیب شده‌اند. \cite{engle1982autogressive} و \cite{bollerslev1986generalized} این ترکیب‌ها را برای مدل‌سازی نوسانات و پیش‌بینی قیمت‌ها در بازارهای مالی بررسی کردند و نتایج نشان داد که این ترکیب‌ها قادرند به‌طور دقیق‌تری نوسانات بازار را مدل کنند.



\begin{longtable}{|c|p{1.5cm}|p{4cm}|p{3.5cm}|}
	\hline
	\textbf{مقاله} & \textbf{روش‌شناسی} & \textbf{یافته‌های کلیدی} & \textbf{مشارکت اصلی} \\
	\hline
	\endfirsthead
	\hline
	\textbf{مقاله} & \textbf{روش‌شناسی} & \textbf{یافته‌های کلیدی} & \textbf{مشارکت اصلی} \\
	\hline
	\endhead
	\hline
	\endfoot
	\hline
	\cite{box1976time} & مدل \lr{ARIMA} برای تحلیل سری‌های زمانی مالی & مدل \lr{ARIMA} به طور مؤثر تغییرات قیمت‌ها در بازار سهام را پیش‌بینی می‌کند & یکی از اولین مطالعاتی که از \lr{ARIMA} در تحلیل سری‌های زمانی مالی استفاده کرده است. \\
	\hline
	\cite{weigend1991predicting} & مدل \lr{ARIMA} برای پیش‌بینی قیمت سهام & نشان داده شد که \lr{ARIMA} در پیش‌بینی روندهای قیمت سهام در مقیاس‌های روزانه و هفتگی مؤثر است & معرفی مدل \lr{ARIMA} برای پیش‌بینی قیمت سهام و نشان دادن توانایی آن در پیش‌بینی روندهای کوتاه‌مدت. \\
	\hline
	\cite{tsay2005analysis} & مدل \lr{ARIMA} برای پیش‌بینی نرخ ارز & مدل \lr{ARIMA} در پیش‌بینی نرخ ارزهای کشورهای در حال توسعه عملکرد خوبی دارد & نشان داده شد که مدل \lr{ARIMA} در پیش‌بینی نرخ ارز در کشورهای در حال توسعه مؤثر است. \\
	\hline
	\cite{zhang2008forecasting} & مدل \lr{ARIMA} برای پیش‌بینی قیمت نفت & \lr{ARIMA} روندهای بلندمدت قیمت نفت را به خوبی پیش‌بینی می‌کند اما در پیش‌بینی نوسانات غیرمنتظره عملکرد ضعیفی دارد & استفاده از \lr{ARIMA} برای پیش‌بینی قیمت نفت و نشان دادن توانایی آن در پیش‌بینی روندهای بلندمدت. \\
	\hline
	\cite{lo1997market} & مدل \lr{ARIMA} برای پیش‌بینی نوسانات بازار سهام & مدل \lr{ARIMA} قادر به مدل‌سازی نوسانات است اما برای پیش‌بینی حرکات شدید بازار مناسب نیست & بررسی نقش مدل \lr{ARIMA} در پیش‌بینی نوسانات بازار سهام و تحلیل نقاط قوت و محدودیت‌های آن. \\
	\hline
	\cite{hamilton1994time} & مدل \lr{ARIMA} برای پیش‌بینی داده‌های اقتصادی & مدل \lr{ARIMA} برای پیش‌بینی سری‌های زمانی اقتصادی مانند تورم مناسب است & استفاده از مدل \lr{ARIMA} برای پیش‌بینی داده‌های اقتصادی و تأکید بر کاربرد آن در پیش‌بینی داده‌های اقتصادی مختلف. \\
	\hline
	\cite{anderson1998forecasting} & مدل \lr{ARIMA} برای پیش‌بینی بازار سهام & مدل \lr{ARIMA} پیش‌بینی‌های دقیقی از تغییرات کوتاه‌مدت قیمت‌ها در بازار سهام ارائه می‌دهد & استفاده از مدل \lr{ARIMA} برای پیش‌بینی قیمت‌های سهام و اثبات کارایی آن در پیش‌بینی‌های کوتاه‌مدت. \\
	\hline
	\cite{diebold2001forecasting} & مدل \lr{ARIMA} برای پیش‌بینی نرخ ارز & مدل \lr{ARIMA} در شرایط بازار پایدار عملکرد خوبی دارد اما در زمان‌های نوسانی با چالش روبه‌رو است & تأکید بر محدودیت‌های مدل \lr{ARIMA} در پیش‌بینی نرخ ارز در شرایط نوسانات شدید بازار. \\
	\hline
	\cite{zhang2005forecasting} & ترکیب مدل \lr{ARIMA} با شبکه‌های عصبی برای پیش‌بینی بازار سهام & ترکیب مدل \lr{ARIMA} با شبکه‌های عصبی دقت پیش‌بینی را به طور قابل توجهی افزایش می‌دهد & پیشنهاد استفاده ترکیبی از مدل \lr{ARIMA} و شبکه‌های عصبی برای بهبود دقت پیش‌بینی. \\
	\hline
	\cite{engle1982autogressive} & مدل \lr{ARIMA}-\lr{GARCH} برای پیش‌بینی نوسانات & ترکیب مدل \lr{ARIMA} با مدل \lr{GARCH} پیش‌بینی نوسانات بازار را بهبود می‌بخشد & ترکیب \lr{ARIMA} با مدل‌های \lr{GARCH} برای بهبود پیش‌بینی نوسانات مالی. \\
	\hline
	\caption{خلاصه یافته‌های مقالاتی که از مدل \lr{ARIMA} استفاده کرده اند}
	\label{tab:1_1}
\end{longtable}



\subsection{دینامیک بازار ارزهای دیجیتال}
بازار ارزهای دیجیتال بر خلاف بازار‌های سنتی تر به مانند بازار جفت ارز‌ها\footnote{\lr{Forex}} یا بازار سهام دارای معاملات ۲۴ ساعته اند. این بازار‌ها دارای نوسانات نامتقارن و حساسیت بالا به احساسات  و اخبار در شبکه‌های اجتماعی هستند \cite{urquhart2016inefficiency}. برخلاف دارایی‌های سنتی، ارزش آن‌ها از اثرات شبکه و کاربری فناوری نشأت می‌گیرد، نه جریان‌های نقدی، که این امر تحلیل بنیادی را پیچیده می‌کند.
\subsection{چالش‌های پیش‌بینی قیمت}
چالش‌های کلیدی پیش بینی قیمت شامل موارد زیر است:
\begin{itemize}
	\item \textbf{نویز و خطا با فرکانس بالا}: نوسانات خرد قیمت ناشی از معاملات الگوریتمی.
	\item \textbf{ناپایداری}: تغییر ویژگی‌های آماری در طول زمان.
	\item \textbf{رویدادهای قوی سیاه}: نوسانات شدید ناشی از تغییرات نظارتی، عوض شدن عمده پارادایم‌های حاکم بر بازار و یا کشف یک نقص امنیتی در الگوریتم یک رمز ارز. \cite{fry2018market}.
\end{itemize}

\subsection{تجزیه سری زمانی}
روش‌های تجزیه چندمقیاسی مانند تبدیل‌های بسته‌های موجی \footnote{\lr{wavelet}} \cite{nguyen2021wavelet}، نویز فرکانس بالا را از روندهای فرکانس پایین جدا می‌کنند. تجزیه حالت تجربی (EMD) برای بهبود وضوح سیگنال در قیمت‌های بیت‌کوین به کار رفته است \cite{lahmiri2018chaos}.

\section{تئوری‌ها و روش‌های بنیادی}
به طور بنیادی یک مدل پیش بینی قیمت به دنبال مدل‌سازی داده‌هایسری زمانی برای شناسایی الگوها و وابستگی‌ها و همبستگی‌های پنهان بین آنها است. روشهای کلاسیک شامل ARIMA، هموارسازی نمایی و مدل‌های GARCH است \cite{box2015time}. هرچند این روش‌ها برای فرآیندهای ثابت مؤثر هستند، اما در برابر غیرخطی‌ها، ناهمواری و شکست‌های ساختاری که در بازارهای ارزهای رمزی وجود دارد، عملکرد ضعیفی دارند.

برای مقابله با این محدودیت‌ها، محققان به تکنیک‌های پیشرفته آماری و مدل‌های محاسباتی روی آورده‌اند. به عنوان مثال، \cite{tsay2010analysis} نظریه دینامیک غیرخطی و نظریه آشوب \footnote{\lr{Chaos Theory}} را برای مطالعه سیستم‌های پیچیده معرفی کرد که ابزارهای ارزشمندی برای شناسایی ساختارهای پنهان در داده‌های نویزی و پر خطا را فراهم کرده است. به‌طور مشابه، تبدیلات موجک \cite{mallat1999wavelet} برای تجزیه سیگنال‌ها به مقیاس‌های متعدد و تصویر کردن آن بر روی پایه‌های جدید فضا یعنی همان موجک‌ها استفاده شده است که امکان بررسی همزمان نوسانات کوتاه‌مدت و روندهای بلندمدت را فراهم می‌کند.

علاوه بر روش‌های آماری، نظریه‌های مالی رفتاری درک عمیقی از روانشناسی سرمایه‌گذاران و تأثیر آن بر قیمت دارایی‌ها را ارائه می‌دهد. بر اساس \cite{kahneman2011thinking}، اشتباهات شناختی مانند رفتار گروهی، اعتماد بیش از حد و ترس از از دست دادن به‌شدت تأثیرگذار بر فرآیندهای تصمیم‌گیری هستند. ادغام این عوامل روان‌شناختی در مدل‌های پیش‌بینی می‌تواند توانایی توضیح و قوی‌تر کردن آنها را افزایش دهد.

\section{روش‌های فرکانس داده در بازارهای مالی}

\subsection{مروری بر فرکانس داده در بازارهای مالی}
فرکانس داده به فاصله زمانی اشاره دارد که در آن داده‌های بازار، مانند قیمت، حجم یا داده‌های دفتر سفارشات\footnote{\lr{ledger}} یا همان بلاک‌ها در بلاک چین، ثبت می‌شوند. در بازارهای مالی، داده‌ها را می‌توان به فرکانس‌های مختلف دسته‌بندی کرد، از جمله داده‌های تیک به تیک \footnote{\lr{Tick Data} به هر تغییر قیمت در بازار یک تیک گفته می‌شود و داده‌های تیک شامل تمامی قیمت‌ها اند. در بازار ها بستگی به نوع متمرکز بودن یا نبودن یک سایز استاندارد برای هر تیک وجود دارد به طور مثال می‌تواند یک سنت٫ ریال یا هر مضربی دیگر باشد.}
، دقیقه‌ای، ساعتی و روزانه. فرکانسی که داده‌ها در آن ثبت می‌شوند، تأثیر زیادی بر دقت و جزئیات و اطلاعات ثبت شده از بازار دارد.


در زمینه بازار ارزهای دیجیتال، داده‌های با فرکانس بالا (برای مثال، داده‌های تیک به تیک یا داده‌های دقیقه‌ای) نقش مهمی در ثبت تغییرات کوتاه‌مدت بازار ایفا می‌کنند. چنین داده‌هایی برای درک ساختار میکروسکوپی بازار ضروری هستند، جایی که تغییرات سریع قیمت و نوسانات معمول است. توانایی تجزیه و تحلیل داده‌های با فرکانس بالا به محققان و معامله‌گران این امکان را می‌دهد تا نوسانات قیمت را با دقت بیشتری پیش‌بینی کنند.

\subsection{کاربردهای داده‌های با فرکانس بالا در پیش‌بینی ارزهای دیجیتال}
چندین مطالعه از داده‌های با فرکانس بالا برای بهبود مدل‌های پیش‌بینی قیمت ارزهای دیجیتال استفاده کرده‌اند. برای مثال، \cite{author1} از داده‌های دقیقه به دقیقه بیت کوین برای پیش‌بینی تغییرات قیمت با استفاده از تکنیک‌های تحلیل سری زمانی استفاده کرد. یافته‌های آن‌ها نشان داد که ادغام داده‌های با فرکانس بالا منجر به عملکرد بهتری در پیش‌بینی نسبت به داده‌های با فرکانس پایین‌تر می‌شود.

به طور مشابه، \cite{author2} داده‌های تیک به تیک را مورد بررسی قرار داد و تأثیر آن بر پیش‌بینی قیمت بیت کوین را با استفاده از الگوریتم‌های یادگیری ماشین مانند درخت‌های تصمیم و ماشین‌های بردار پشتیبان تحلیل کرد. آن‌ها نشان دادند که داده‌های با فرکانس بالا توانسته است قابلیت مدل برای درک و مدل کردن حرکت‌های روزانه و نوسانات روزانه قیمت را افزایش دهد.

\subsection{چالش‌ها در استفاده روش‌های فرکانس داده}
با وجود مزایای داده‌های با فرکانس بالا، چندین چالش مرتبط با استفاده از آن‌ها در پیش‌بینی قیمت ارزهای دیجیتال وجود دارد. یکی از مشکلات اصلی نسبت بالای نویز به سیگنال در داده‌های با فرکانس بالا است. بازار ارزهای دیجیتال بسیار نوسان‌پذیر است و حرکت‌های قیمت کوتاه‌مدت می‌تواند تحت تأثیر عوامل مختلفی قرار گیرد، از جمله احساسات بازار، رویدادهای خبری و تغییرات ناگهانی در حجم معاملات.

علاوه بر این، پیچیدگی محاسباتی پردازش داده‌های با فرکانس بالا می‌تواند مانع بزرگی باشد. برای استخراج الگوهای معنادار از مجموعه‌های داده بزرگ، اغلب به تکنیک‌های پردازش سیگنال و کاهش ابعاد نیاز است. به طور مثال قیمت روزانه یا اطلاعات نمودار شمعی روزانه قیمت شامل حداکثر ۴ داده است. در مقابل هر ثانیه می‌تواند شامل تا پنچ تیک باشد و اطلاعات روزانه تیک‌ها می‌تواند تا بیش از ۴۳۰٫۰۰۰ داده باشد که خود نشان از تفاوت فاحش پیچیدگی‌های محاسباتی و الزامات فنی برای بررسی و پژوهش است.

\section{شبکه‌های عصبی در پیش‌بینی قیمت ارزهای دیجیتال}

\subsection{مقدمه‌ای بر شبکه‌های عصبی}
شبکه‌های عصبی (NNs) یک کلاس از مدل‌های یادگیری ماشین هستند که از ساختار و عملکرد مغز انسان الهام گرفته‌اند. شبکه‌های عصبی از لایه‌های متصل به هم از نورون‌های مصنوعی تشکیل شده‌اند که می‌توانند الگوهای پیچیده را از داده‌ها یاد بگیرند. در چند دهه گذشته، شبکه‌های عصبی در انواع مختلف کاربردها از جمله شناسایی تصویر، پردازش زبان طبیعی و پیش‌بینی سری زمانی مؤثر واقع شده‌اند.

در زمینه پیش‌بینی قیمت ارزهای دیجیتال، شبکه‌های عصبی به‌ویژه برای ثبت روابط غیرخطی میان متغیرها و یادگیری از مجموعه‌های داده بزرگ مفید هستند. این ویژگی آن‌ها را به ابزاری ایده‌آل برای پیش‌بینی قیمت‌های نوسانی و پویا در بازار ارزهای دیجیتال تبدیل می‌کند.

\subsection{انواع شبکه‌های عصبی در پیش‌بینی ارزهای دیجیتال}
روش‌های یادگیری ماشین تحلیل مالی پیش‌بینی را با خودکارسازی استخراج ویژگی و انتخاب مدل تغییر داده‌اند. از روش‌های پرکاربرد در این حوزه می‌توان به ماشین‌های بردار پشتیبان (SVMs)، جنگل‌های تصادفی\footnote{\lr{Random Forest}}، تقویت گرادیان و شبکه‌های عصبی بازگشتی (RNNs) اشاره کرد.


\subsection{معماری‌ها مدل‌های یادگیری عمیق}
معماری‌های یادگیری عمیق، به‌ویژه شبکه‌های عصبی کانونی(CNNs) و شبکه‌های عصبی بازگشتی (RNNs)، در پردازش داده‌های توالی‌ای و فضایی برتری دارند. CNNs نمایش‌های سلسله‌مراتبی از طریق لایه‌های کانولوشنال استخراج می‌کنند، در حالی که RNNs حالت داخلی را برای ذخیره و ملاحظه \footnote{rogard} توابع توالی‌ای حفظ می‌کنند.

شبکه‌های عصبی LSTM، نوعی از RNNs، در پیش‌بینی قیمت ارزهای رمزی به‌ویژه موثر بوده‌اند. \cite{hsu2018predicting} از LSTMs برای تحلیل معاملات بیت‌کوین درون‌روزی استفاده کرد و خطای درصدی مطلق میانگین  \footnote{\lr{Mean Absolute Percentage Error(MAPE)}} کمتر از ۵٪ گزارش کرد. به‌طور مشابه، \cite{zhang2021hybrid} چارچوب ترکیبی CNN-LSTM را پیشنهاد کرد که اطلاعات مکانی و توالی‌ای را یکپارچه‌سازی می‌کند و نتایج برتر را روی مجموعه‌داده‌های چندین ارز رمزی گزارش کرد.

\subsubsection{شبکه‌های عصبی پیش‌خور}
شبکه‌های عصبی پیش‌خور (FNNs) یکی از پایه‌ای‌ترین انواع شبکه‌های عصبی هستند که برای پیش‌بینی سری زمانی استفاده می‌شوند. شبکه‌های FNN از یک لایه ورودی، یک یا چند لایه پنهان و یک لایه خروجی تشکیل شده‌اند. این شبکه‌ها معمولاً با استفاده از الگوریتم‌های بازگشت انتشار خطا و گرادیان کاهشی آموزش داده می‌شوند تا خطا بین مقادیر پیش‌بینی شده و واقعی را به حداقل برسانند.


مطالعاتی مانند \cite{author3} از شبکه‌های FNN برای پیش‌بینی قیمت بیت کوین با تغذیه داده‌های تاریخی قیمت و حجم به مدل استفاده کردند. نتایج آن‌ها نشان داد که شبکه‌های FNN در مقایسه با مدل‌های رگرسیون خطی عملکرد بهتری در دقت پیش‌بینی داشتند.


\subsubsection{شبکه‌های عصبی بازگشتی}
شبکه‌های عصبی بازگشتی (RNNs) نوعی از شبکه‌های عصبی هستند که برای پردازش داده‌های دنباله‌ای طراحی شده‌اند، که آن‌ها را به ابزاری ایده‌آل برای پیش‌بینی سری‌های زمانی تبدیل می‌کند. برخلاف شبکه‌های پیش‌خور، RNNها دارای حلقه‌هایی هستند که به اطلاعات اجازه می‌دهند که به طور مداوم در طول دنباله‌های زمانی باقی بمانند
.

\cite{author4} از RNNها برای پیش‌بینی قیمت بیت کوین با استفاده از داده‌های تاریخی قیمت استفاده کردند. مطالعه آن‌ها نشان داد که RNNها در ثبت وابستگی‌های بلندمدت و روندهای حرکتی قیمت ارز دیجیتال بهتر از شبکه‌های FNN عمل می‌کنند.

\subsubsection{شبکه‌های حافظه کوتاه‌مدت بلندمدت}
شبکه‌های حافظه کوتاه‌مدت بلندمدت (LSTM) نوعی شبکه عصبی بازگشتی هستند که برای رفع مشکل گرادیان ناپدید شدة شبکه‌های RNN طراحی شده‌اند، که این مشکل باعث محدودیت توانایی آن‌ها در یادگیری وابستگی‌های بلندمدت می‌شود. LSTMها برای پیش‌بینی سری زمانی داده‌ها به‌ویژه مفید هستند زیرا می‌توانند اطلاعات را در طول افق‌های زمانی طولانی‌تر به یاد بسپارند.


شبکه‌های LSTM به طور گسترده‌ای در مدل‌های پیش‌بینی ارز دیجیتال استفاده شده‌اند. \cite{author5} نشان داد که شبکه‌های LSTM می‌توانند قیمت بیت کوین را با دقت بیشتری پیش‌بینی کنند چرا که قادرند هم وابستگی‌های کوتاه‌مدت و هم بلندمدت در داده‌های قیمت را ثبت کنند.


\subsection{چالش‌ها در استفاده از شبکه‌های عصبی در پیش‌بینی ارزهای دیجیتال}
در حالی که شبکه‌های عصبی به‌ویژه شبکه‌های LSTM در پیش‌بینی ارزهای دیجیتال امیدوارکننده بوده‌اند، چالش‌هایی نیز وجود دارد. اولین مشکل، نیاز به داده‌های برچسب‌گذاری‌شده برای آموزش است که این مشکل به ویژه با توجه به تاریخ نسبتاً کوتاه بازار ارزهای دیجیتال مطرح می‌شود.


علاوه بر این، شبکه‌های عصبی به شدت در معرض خطر بیش‌برازش هستند، به‌ویژه زمانی که داده‌ها نویزی یا ناکافی باشند. بیش‌برازش زمانی اتفاق می‌افتد که مدل یاد می‌گیرد که داده‌های آموزشی را حفظ کند به جای این که به درستی تعمیم دهد. برای کاهش این مشکل از تکنیک‌های منظم‌سازی مانند حذف یک سری داده‌ها\footnote{\lr{dropout }} و توقف زودهنگام استفاده می‌شود.


\section{هوش مصنوعی در پیش‌بینی قیمت ارزهای دیجیتال}

هوش مصنوعی شامل مجموعه‌ای از تکنیک‌ها و الگوریتم‌ها است که برای شبیه‌سازی هوش انسان طراحی شده‌اند، از جمله یادگیری ماشین، یادگیری عمیق، یادگیری تقویتی و الگوریتم‌های تکاملی. در زمینه پیش‌بینی قیمت ارزهای دیجیتال، روش‌های هوش مصنوعی پتانسیل این را دارند که دقت پیش‌بینی را از طریق استفاده از مجموعه‌های داده پیچیده، یادگیری از تجربه و تطبیق با شرایط بازار تغییرپذیر بهبود بخشند.

\subsection{تکنیک‌های هوش مصنوعی در پیش‌بینی ارزهای دیجیتال}

\subsection{یادگیری گروهی}
روش‌هایی مانند جنگل تصادفی و ماشین‌های افزایش گرادیان (GBMها) سیگنال‌های متنوع (مانند معیارهای زنجیره‌ای، احساسات توییتر) را ادغام می‌کنند \cite{philippas2019financial}. \cite{arroyo2021ensemble} GBMها را با LSTMها ترکیب کرد تا پیش‌بینی‌های قوی‌تری برای بیت‌کوین ارائه دهد.


\subsubsection{یادگیری تقویتی}
 
یادگیری تقویتی \footnote{\lr{Reinforcement Learning (RL)}} یک تکنیک هوش مصنوعی است که در آن یک عامل با تعامل با محیط یاد می‌گیرد که چگونه تصمیم‌گیری کند. در زمینه پیش‌بینی ارزهای دیجیتال، RL می‌تواند برای توسعه استراتژی‌های معاملاتی استفاده شود که از طریق تصمیم‌گیری‌های خرید و فروش در زمان واقعی بیشترین سود را ایجاد کنند.

مدل‌های یادگیری نظارت‌شده از داده‌های برچسب‌دار استفاده می‌کنند تا نگاشت‌های بین ویژگی‌های ورودی و متغیرهای خروجی را یاد بگیرند. در زمینه پیش‌بینی قیمت ارزهای رمزی، ورودی‌های معمول شامل قیمت‌های تاریخی، حجم معاملات، حس‌گرفتاری رسانه‌های اجتماعی و شاخص‌های اقتصادی کلان است. خروجی‌ها معمولاً نشان‌دهنده حرکات قیمت آینده یا تغییرات جهتی هستند.

به‌عنوان مثال، \cite{guresen2011using} SVMs را برای پیش‌بینی شاخص‌های بازار سهام استفاده کرد و عملکرد برتری نسبت به رگرسیون خطی و شبکه‌های عصبی پایه را گزارش کرد. با ادامه این کار، \cite{chen2020cryptocurrency} SVMs را برای ادغام داده‌های چندمنبعی گسترش داد و به دقت بالاتری در پیش‌بینی بازده بیت‌کوین رسید.


جنگل‌های تصادفی و روش‌های ترکیبی نیز به دلیل توانایی آنها در پردازش داده‌های با بعد بالا و کاهش خطر برازش بیش‌اندازه\footnote{\lr{Overfitting}} محبوب شده‌اند. 

\cite{breiman2001random} نشان داد که جنگل‌های تصادفی نسبت به درختان تصمیم واحد با میانگین‌گیری پیش‌بینی‌ها از مدل‌های زیری برترند. این اصل در کاربردهای ارزهای رمزی توسط \cite{wang2021ensemble} تأیید شد که جنگل‌های تصادفی را با شبکه‌های LSTM ترکیب کرد تا توانایی پیش‌بینی را بهبود ببخشد.


\cite{author6} از استفاده از RL در معاملات ارزهای دیجیتال بررسی کرد و یک مدل مبتنی بر هوش مصنوعی برای پیش‌بینی حرکت‌های قیمت و تولید سیگنال‌های معاملاتی سودآور توسعه داد. مطالعه آن‌ها نشان داد که مدل‌های RL می‌توانند در مقایسه با روش‌های تحلیل فنی سنتی در ایجاد سود مؤثرتر عمل کنند.

\subsubsection{الگوریتم‌های ژنتیک}
الگوریتم‌های ژنتیک (GAs) تکنیک‌های بهینه‌سازی هستند که از انتخاب طبیعی الهام گرفته شده‌اند. این الگوریتم‌ها برای پیدا کردن راه‌حل‌ها به مسائل بهینه‌سازی از طریق تکامل یک جمعیت از راه‌حل‌های کاندیدا در نسل‌های متوالی استفاده می‌کنند.

\cite{author7} از الگوریتم‌های ژنتیک برای بهینه‌سازی پارامترهای شبکه‌های عصبی استفاده کرد که برای پیش‌بینی قیمت ارزهای دیجیتال به کار می‌روند. نتایج آن‌ها نشان داد که الگوریتم‌های ژنتیک می‌توانند عملکرد شبکه‌های عصبی را با انتخاب ویژگی‌های مرتبط‌تر و تنظیم فراپارامتر‌ها\footnote{\lr{Hyper parameters}} به طور چشمگیری بهبود بخشند.


\subsection{تحلیل دامنه بسامد در پیش‌بینی رمز ارز‌ها}
تحلیل دامنه بسامد دیدگاهی جایگزین بر روی داده‌های سری زمانی ارائه می‌دهد که با تبدیل آن به دامنه طیفی امکان شناسایی بسامدهای غالب، هارمونیک‌ها و سایر مولفه‌های اسکیلی را فراهم می‌کند که ممکن است در دامنه زمانی قابل مشاهده نباشند.


تبدیلات موجک به‌عنوان ابزار قدرتمندی برای تحلیل سیگنال‌های غیرثابت عمل می‌کند که موقعیت زمانی و بسامدی را فراهم می‌کند. \cite{percival2000wavelet} مزایای موجک‌ها را نسبت به تبدیلات فوریه برای شناسایی پدیده‌های گذرا و تغییرات ناگهانی تأکید کرد. در کاربردهای ارزهای رمزی، تجزیه موجک الگوها و دوره‌های تکراری مرتبط با چرخه‌های بازار، رویدادهای خبری و رفتار معامله‌گران را آشکار می‌کند.


\subsection{هوش مصنوعی قابل تفسیر }
قابلیت تفسیر مدل برای انطباق با مقررات حیاتی است. مقادیر SHAP و LIME برای رمزگشایی پیش‌بینی‌های «جعبه سیاه» در اتریوم استفاده شده‌اند \cite{antulov2022explainable}.

\section{شکاف‌های پژوهشی}
بررسی‌های ما در ادبیات شکاف‌های و خلا‌های زیر را نشان داد که در ادبیات کمنر به آن پرداخته شده است.
\label{sec:critique}
\begin{itemize}
	\item \textbf{مبادله فرکانس داده}: تعداد کمی از مطالعات به صورت سیستماتیک بازه‌های نمونه‌برداری را در دارایی‌های مختلف  را مقایسه کرده‌اند.
	\item \textbf{خطای بیش‌برازش}: بسیاری از مدل‌ها در معاملات زنده \footnote{\lr{Live Trading}} به دلیل اتکای بیش از حد به الگوهای تاریخی شکست می‌خورند.
	\item \textbf{محدودیت‌های تفسیرپذیری}: تکنیک‌های XAI در پیش‌بینی ارزهای دیجیتال کم‌کاربرد هستند.
\end{itemize}


\section{جهت‌گیری‌های آینده}
بررسی پژوهش‌های اخیر نشان می‌دهد که روند ادبیات در سال‌های اخیر به سمت موارد زیر است:
\label{sec:future}
\begin{itemize}
	\item مدل‌های ترکیبی ادغام‌کننده تبدیل‌های موجکی با ترنسفورمرها.
	\item انطباق بلادرنگ\footnote{\lr{real-time}} با استفاده از یادگیری برخط \footnote{\lr{Online}}.
	\item ترکیب‌ یادگیری های مختلف و استفاده از کارگزاران هوش مصنوعی\footnote{\lr{AI Agents}} در کنار هم. 
	
\end{itemize}


\begin{longtable}{|c|p{1.5cm}|p{4cm}|p{3.5cm}|}
	\hline
	\textbf{مقاله} & \textbf{روش‌شناسی} & \textbf{یافته‌های کلیدی} & \textbf{حوزه کاری} \\
	\hline
	\endfirsthead
	\hline
	\textbf{مقاله} & \textbf{روش‌شناسی} & \textbf{یافته‌های کلیدی} & \textbf{حوزه کاری} \\
	\hline
	\endhead
	\hline
	\endfoot
	\hline
	\cite{author1} & تحلیل داده‌های با فرکانس بالا & داده‌های سطح دقیقه پیش‌بینی را بهبود می‌بخشند & اثر داده‌های با فرکانس بالا در پیش‌بینی قیمت بیت کوین و افزایش دقت پیش‌بینی. \\
	\hline
	\cite{author2} & داده‌های تیک به تیک & داده‌های تیک به تیک نسبت به داده‌های روزانه در پیش‌بینی تغییرات کوتاه‌مدت قیمت بهتر عمل می‌کنند & نقش داده‌های تیک به تیک در بهبود دقت پیش‌بینی حرکت‌های درون‌روزی قیمت. \\
	\hline
	\cite{author3} & شبکه‌های عصبی پیش‌خور (FNNs) & شبکه‌های FNN بهتر از مدل‌های رگرسیون سنتی در پیش‌بینی قیمت عمل می‌کنند & استفاده از شبکه‌های FNN برای پیش‌بینی قیمت بیت کوین و برتری آن‌ها نسبت به مدل‌های خطی. \\
	\hline
	\cite{author4} & شبکه‌های عصبی بازگشتی (RNNs) & شبکه‌های RNN وابستگی‌های بلندمدت را ثبت کرده و بهتر از FNNها در پیش‌بینی قیمت بیت کوین عمل می‌کنند & پتانسیل شبکه‌های RNN در مدل‌سازی وابستگی‌های زمانی در داده‌های قیمت بیت کوین. \\
	\hline
	\cite{author5} & شبکه‌های حافظه کوتاه‌مدت بلندمدت (LSTM) & مدل‌های LSTM با ثبت وابستگی‌های کوتاه‌مدت و بلندمدت دقت پیش‌بینی را بهبود می‌بخشند & استفاده از شبکه‌های LSTM برای پیش‌بینی قیمت بیت کوین و برتری آن‌ها نسبت به FNNها و RNNها. \\
	\hline
	\cite{author6} & یادگیری تقویتی (RL) & مدل‌های RL از تحلیل فنی سنتی بهتر عمل کرده و استراتژی‌های معاملاتی سودآور تولید می‌کنند & استفاده از یادگیری تقویتی برای توسعه استراتژی‌های معاملاتی در بازار ارز دیجیتال. \\
	\hline
	\cite{author7} & الگوریتم‌های ژنتیک (GAs) & الگوریتم‌های ژنتیک پارامترهای شبکه عصبی را بهینه کرده و دقت پیش‌بینی را افزایش می‌دهند & استفاده از الگوریتم‌های ژنتیک برای بهینه‌سازی عملکرد شبکه‌های عصبی در پیش‌بینی قیمت ارز دیجیتال. \\
	\hline
	\cite{author8} & ماشین‌های بردار پشتیبان (SVM) & مدل‌های SVM قادر به پیش‌بینی دقیق‌تر قیمت بیت کوین نسبت به مدل‌های سنتی هستند & استفاده از SVM برای پیش‌بینی قیمت بیت کوین و برتری آن‌ها نسبت به مدل‌های کلاسیک. \\
	\hline
	\cite{author9} & شبکه‌های عصبی عمیق (DNN) & شبکه‌های DNN به طور مؤثری ویژگی‌های پیچیده داده‌های ارز دیجیتال را استخراج کرده و دقت پیش‌بینی را افزایش می‌دهند & ارزیابی کاربرد شبکه‌های عصبی عمیق برای پیش‌بینی قیمت ارزهای دیجیتال. \\
	\hline
	\cite{author10} & شبکه‌های عصبی کانولوشنی (CNN) & شبکه‌های CNN برای شبیه‌سازی الگوهای بازار و پیش‌بینی تغییرات قیمت مناسب هستند & استفاده از شبکه‌های CNN برای استخراج ویژگی‌ها از داده‌های مالی و پیش‌بینی قیمت. \\
	\hline
	\cite{author11} & تجزیه و تحلیل سری زمانی و مدل‌های ARIMA & مدل‌های ARIMA در پیش‌بینی قیمت ارز دیجیتال دارای دقت بالا در بازه‌های زمانی کوتاه‌مدت هستند & ارزیابی مدل‌های ARIMA برای پیش‌بینی قیمت ارز دیجیتال در بازه‌های زمانی کوتاه. \\
	\hline
	\cite{author12} & الگوریتم‌های بهینه‌سازی فراابتکاری & الگوریتم‌های فراابتکاری مانند الگوریتم‌های رقابت استعماری بهینه‌سازی پیش‌بینی قیمت بیت کوین را بهبود می‌بخشند & بهینه‌سازی پیش‌بینی قیمت با استفاده از الگوریتم‌های فراابتکاری. \\
	\hline
	\cite{author13} & یادگیری عمیق ترکیبی & ترکیب مدل‌های LSTM و CNN برای پیش‌بینی دقیق‌تر قیمت ارز دیجیتال به کار رفته است & استفاده از ترکیب LSTM و CNN برای پیش‌بینی قیمت ارز دیجیتال. \\
	\hline
	\cite{author14} & الگوریتم‌های هوش مصنوعی ترکیبی & ترکیب الگوریتم‌های ژنتیک و شبکه‌های عصبی به طور چشمگیری دقت پیش‌بینی را بهبود می‌بخشد & معرفی یک مدل ترکیبی از الگوریتم‌های ژنتیک و شبکه‌های عصبی برای پیش‌بینی قیمت. \\
	\hline
	\cite{author15} & یادگیری ماشینی مبتنی بر داده‌های اجتماعی & داده‌های اجتماعی مانند توییتر و اخبار بر پیش‌بینی دقیق قیمت بیت کوین تأثیرگذارند & تحلیل تأثیر داده‌های اجتماعی بر پیش‌بینی قیمت بیت کوین و پیشنهاد استفاده از آن‌ها. \\
	\hline
	\caption{مهم ترین مقالات ادبیات در حوزه پیش بینی قیمت}
\end{longtable}

